\documentclass[10]{article}

% standard packages

% A more pleasant font
\usepackage[T1]{fontenc} % use postscript type 1 fonts
\usepackage{textcomp} % use symbols in TS1 encoding
\usepackage{mathptmx,helvet,courier} % use nice, standard fonts for roman, sans and monospace respectively

% Improves the text layout
\usepackage{microtype}

\usepackage{lscape}
\usepackage{fancyhdr}
\usepackage{epsfig}
\usepackage{subfigure}
\usepackage{url}
\usepackage{graphics}
\usepackage{enumerate}
\usepackage{ifthen}
\usepackage{float}
\usepackage{listings}
\lstset{basicstyle=\ttfamily}
% \usepackage[strings]{underscore}
% \usepackage{underscore}
\usepackage[bookmarks=true,bookmarksnumbered=true,pdfborder={0 0 0}]{hyperref}

\lstset{
  literate={\_}{}{0\discretionary{\_}{}{\_}}%
}

\usepackage[table]{xcolor}
\usepackage{booktabs}

\DeclareUrlCommand\email{}

\pagestyle{fancy}
\rhead{}

\newfloat{listing}{tbp}{lol}
\floatname{listing}{Listing}

\begin{document}
\title{How to Avoid OOTA Without Really Trying}

\newcommand{\co}[1]{\lstinline[breaklines=yes,breakatwhitespace=yes]{#1}}

% Blind review process.
% \author{
% Alan Stern\\{\small \email{stern@rowland.harvard.edu}} \and
% Paul E.~McKenney\\{\small \email{paulmck@kernel.org}} \and
% Michael Wong\\{\small \email{fraggamuffin@gmail.com}} \and
% Maged Michael\\{\small \email{maged.michael@gmail.com}}
% }
\maketitle{}

\begin{abstract}
	The out-of-thin-air (OOTA) properties
	of \co{memory_order_relaxed} in the C++ memory model
	have caused considerable consternation over the years.
	Attempts to create memory models that rule out OOTA behaviors
	have been either non-executable, complex, or unloved by C++
	implementers.
	But at the same time, we know of no instances of OOTA behavior
	in real C++ implementations.

	We focus on shared-memory programs and
	C++ implementations based on traditional compilers and
	computing hardware, including CPUs and GPGPUs.
	This context permits us to consider the detailed relations
	between source code and machine code required by the restricted
	nature of volatile atomic accesses.
	The OOTA problem and the related challenge of coming to grips
	with semantic dependency are much more tractable at the hardware
	level than at the source level, thanks to existing formal
	hardware models.
	We show that these models' constraints prevent OOTA cycles
	from occurring in undefined-behavior-free C++ programs running
	on compiler-based implementations, provided the cycles involve
	only volatile atomics.
	We also extend this work to nonvolatile atomics by defining
	``quasi-volatile'' behavior that we expect C++ implementations
	will adhere to if they perform single-thread analysis.
\end{abstract}

\section{Introduction and Background}
\label{sec:Introduction}

This paper shows that compiler-based C++ implementations subject to
reasonable constraints on how they treat accesses to atomic objects
cannot exhibit out-of-thin-air (OOTA) cycles.
It follows that these implementations need to take almost no special
actions to avoid OOTA.
In fact, for many compilers the constraints boil down to a simple
``Don't invent or duplicate atomic accesses'', which the compiler
probably wouldn't do anyway.
Absolutely no changes are required to either the standard or to user code.

We begin with a brief overview of the OOTA problem followed by an equally
brief summary of prior work in this area.

\subsection{Brief OOTA Overview}
\label{sec:Brief OOTA Overview}

In broad terms,
OOTA occurs theoretically when a group of threads load from each others' stores
and each thread's store depends on the value returned by that thread's load.
The collection of loads and stores forms an \emph{OOTA cycle}.
In the most extreme cases a nonsensical value can pop up ``out of thin air'';
however, as shown by
Listing~\ref{lst:OOTA Cycle} below,
OOTA cycles need not involve nonsensical values.

\subsubsection{Simple OOTA Cycle}
\label{sec:Simple OOTA Cycle}

\begin{listing}[tbp]
% Constructed by hand.
{
\scriptsize
\begin{tabular}{r|l|l}
& \multicolumn{1}{c}{P0} & \multicolumn{1}{c}{P1} \\
\hline
1. &
\texttt{~~int r1 =\textsubscript{rlx} x} &
	\texttt{~~int r2 =\textsubscript{rlx} y} \\
2. &
\texttt{~~y =\textsubscript{rlx} r1} &
	\texttt{~~x =\textsubscript{rlx} r2} \\
\end{tabular}

\vspace{0.1in}
Condition (0:r1=42 $\wedge$ 1:r2=42)
sometimes satisfied.
}
\caption{Simple OOTA}
\label{lst:Simple OOTA}
\end{listing}

Listing~\ref{lst:Simple OOTA} shows a simple example where an OOTA
cycle might result in all of \co{x}, \co{y}, \co{r1}, and \co{r2}
having final values of 42, despite the fact that there is nothing in
the default-zero initial values or the executable code to support such
an outcome~\cite{PaulEMcKenney2020RelaxedGuideRelaxed}.
The ``=\textsubscript{rlx}'' denotes a relaxed access, with variables
whose names are ``\co{r}'' followed by a number denoting registers,
and other names denoting shared variables.
First, P0's line~1 does a relaxed atomic load from \co{x} into \co{r1},
reading the value of P1's line-2 store rather than \co{x}'s initial
value of zero and somehow obtaining 42.
Second, P0's line~2 stores \co{r1}, and thus 42, to \co{y}.
Third, P1's line~1 loads 42 from \co{y} to \co{r2}.
Fourth and finally, P1's line~2 stores 42 to \co{x}, justifying the value
loaded by P0's line~1.

Because nothing else in the C++ memory model rules out such OOTA cycles
(as indicated by the ``sometimes satisfied'' in the figure), the C++
standard explicitly prohibits them in 33.5.4p8
(\co{[atomics.order]})~\cite{ThomasKoeppe2023N4950}:
\begin{quote}
	Implementations should ensure that no “out-of-thin-air” values
	are computed that circularly depend on their own computation.
\end{quote}
The standard's prohibition of OOTA is of course important, for example, to
prevent misapplication of the as-if rule in oracular C++ implementations,
but those of us writing code in the real world must rely on actual C++
implementations.
And in these implementations, this prohibition is in fact enforced by TSO
ordering in strongly ordered systems and by data-dependency ordering in
weakly ordered systems.\footnote{
	The need to prohibit simple OOTA is one reason why compiler-based
	value speculation optimizations require checks on such
	speculation, and these checks must be based on actual values
	loaded.}

In Listing~\ref{lst:Simple OOTA}
there is a \emph{semantic dependency} from line~1 to line~2 in both
P0 and P1.
(Roughly speaking, there is a semantic dependency from a given load
to a given store when \emph{all other things being equal, a change in the
value loaded can change the value stored or prevent the store from
occurring at all.}
Here the dependencies are trivial because the values stored simply
\emph{are} the values that were loaded.)
Since real-world CPUs cannot store something
until they have determined its value,\footnote{
	Another way of saying this is that real-world CPUs do not
	make their stores visible to other CPUs until those stores
	are no longer speculative.
	See Section~\ref{sec:Hardware Architecture and Design}
	for more discussion about hardware speculation.}
the stores on line~2 cannot take place until the P0's and P1's CPUs
know what values are returned by the loads on line~1.
Thus the hardware orders these stores after their corresponding loads,
and this ordering prevents the OOTA result.

\subsubsection{Simple Reordering}
\label{sec:Simple Reordering}

\begin{listing}[tbp]
% Constructed by hand.
{
\scriptsize
\begin{tabular}{r|l|l}
& \multicolumn{1}{c}{P0} & \multicolumn{1}{c}{P1} \\
\hline
1. &
\texttt{~~int r1 =\textsubscript{rlx} x} &
	\texttt{~~int r2 =\textsubscript{rlx} y} \\
2. &
\texttt{~~y =\textsubscript{rlx} r1} &
	\texttt{~~x =\textsubscript{rlx} 42} \\
\end{tabular}

\vspace{0.1in}
Condition (0:r1=42 $\wedge$ 1:r2=42)
sometimes satisfied.
}
\caption{Simple Reordering}
\label{lst:Simple Reordering}
\end{listing}

It is important to distinguish true OOTA cycles from OOTA-like
behavior caused by simple reordering.
An example of simple reordering is shown in
Listing~\ref{lst:Simple Reordering}~\cite{PaulEMcKenney2020RelaxedGuideRelaxed}.
Both the C++ compiler and the CPU are within their rights to reorder P1's
lines~1 and~2, which can result in all of \co{x}, \co{y}, \co{r1},
and \co{r2} having the value 42 via the following steps.
First, P1's line~2 stores 42 to \co{x}.
Second, P0's line~1 loads 42 from \co{x} into \co{r1}.
Third, P0's line~2 stores \co{r1}, and thus 42, to \co{y}.
Fourth and finally, P1's line~1 loads 42 from \co{y} to \co{r2}.
Current C++ implementations can and do exhibit this reordering behavior,
as indicated by the ``sometimes satisfied'' in the figure.

\subsubsection{Non-Trivial OOTA}
\label{sec:Non-Trivial OOTA}

This paper will follow P2055R0~\cite{PaulEMcKenney2020RelaxedGuideRelaxed}
in using the term \emph{full C++} to denote the standard including the
prohibition of OOTA mentioned above.
Unlike that article, we will use the term \emph{loose C++}
(rather than \emph{strict C++}, which seems too similar to
\emph{full C++} for comfort)
to denote a hypothetical standard that excludes this prohibition but
is otherwise identical to full C++.
Unqualified \emph{C++} means full C++.

\begin{listing}[tbp]
@@ DisplayRunLitmus litmus/oota-ctrl.litmus @@
\caption{OOTA Cycle}
\label{lst:OOTA Cycle}
\end{listing}

Listing~\ref{lst:OOTA Cycle}
shows a code fragment that under loose C++ has an OOTA cycle, though
the cycle is of course prohibited in full C++.
This example differs from that of Section~\ref{sec:Simple OOTA Cycle}
in that line~3 of both processes stores the value 42, but only if
line~2 of a given process determines that the value loaded was 42,
keeping in mind that variables are default-initialized to zero.

This OOTA cycle is constructed in the following manner.
First, P0's and P1's line~1 load from \co{y} and \co{x}, respectively,
both somehow returning the value 42.
Second, P0's and P1's line~2 determine that the values loaded were both
42, so each process advances to its line~3.
Third, P0's and P1's line~3 stores the value 42 that was (somehow)
clairvoyantly loaded by the other process's line~1.

The next section looks at how prior work has refined these issues.

\subsection{Prior Work}
\label{sec:Prior Work}

All OOTA workers owe a debt to the foundational work in the infamous
``Causality Test Cases''.\footnote{
	\url{http://www.cs.umd.edu/~pugh/java/memoryModel/unifiedProposal/testcases.html}.}
% %%% Eventually cite P3064.

Some executable C++ memory models correctly flag at least some executions
involving OOTA cycles~\cite{JadeAlglave2014HerdingCats}.\footnote{
	Others cleverly avoid this issue by forbidding atomic
	stores of nonconstant values~\cite{MarkBatty2011cppmem}.}
% @@@ Better herd7 C11 citation?
However, because these models are atemporal, they cannot reject
OOTA executions other than by flagging the OOTA value as arbitrary,
which some in fact do in at least some cases.

P0442R0 (``Out-of-Thin-Air Execution is Vacuous'')~\cite{PaulEMcKenney2016OOTA}
provided a decision procedure for classifying behaviors as permitted
misordering on the one hand or disallowed OOTA on the other, using
a perturbation method based on the insight that all OOTA behaviors are
fixed-point computations.

Some workers recommend avoiding OOTA by forcing prior relaxed
loads to be ordered before subsequent relaxed
stores~\cite{Boehm:2014:OGA:2618128.2618134,HansBoehm2019OOTArevisitedAgain,Lahav:2017:RSC:3062341.3062352},
but this can require real instructions be
executed~\cite[Section 7.1]{Maranget2012TutorialARMPower},
consuming real time and real electrical power to solve a strictly
theoretical problem.
This might have been acceptable in the 1960s of some of the authors'
youths, but it is now the year 2024.

Other workers recommend various procedures to identify and avoid OOTA
cycles~\cite{Lahav:2017:RSC:3062341.3062352,Sinclair:2017:CAR:3079856.3080206,Lee:10.1145/3385412.3386010,MarkBatty2019ModularRelaxedDependenciesOOTA},
but none of these have been looked upon favorably by C++ implementers.
Some of these workers appear to have abandoned this effort, but as of
2024, Mark Batty is persisting with modular relaxed dependencies.
@@@ cite web site.

Goldblatt looked at interactions between OOTA cycles and
undefined behavior (UB)~\cite{DavidGoldblatt2019NoElegantOOTAfix}.
% Appendix~\ref{app:Aside on Undefined Behavior}
% analyzes his examples and notes ways to separate UB and OOTA-cycle
% concerns.
This document will concentrate on examples lacking UB.

All this work focused on either identifying OOTA or seeing how C++
implementations could avoid it.
None applied real-world hardware ordering constraints to the problem
of avoiding OOTA cycles,
yet doing so might help explain why no known real-world C++ implementation
results in OOTA behavior.
We therefore dig more deeply into OOTA cycles in the light
of these constraints.

\section{OOTA and Semantic Dependencies}
\label{sec:OOTA and Semantic Dependencies}

Section~\ref{sec:OOTA: rf versus rfe}
demonstrates advantages of formulating an OOTA cycle as a cycle in
sdep $\cup$ rfe instead of the traditional sdep $\cup$ rf.
Section~\ref{sec:Properties of Semantic Dependencies}
then discusses properties of semantic dependences, showing that they
are functions of executions rather than strictly of source code,
among other things.

\subsection{OOTA: rf versus rfe}
\label{sec:OOTA: rf versus rfe}

Semantic dependencies form only one type of link in an OOTA cycle.
The other type extends from a given store to a load that returns the
value stored.
It is tempting to argue that the store must precede
the load in global time and combine this with the intuitive notion that
any real C++ implementation must consume global time when computing a
semantic dependency.
This combination suggests that OOTA cycles cannot occur.
The idea has been formalized by defining an OOTA cycle as a cycle
in sdep $\cup$ rf~\cite{PaulEMcKenney2014OOTA},
% This citation credits the formulation to Ali Sezgin, third author
% of N4323 ("Out-of-Thin-Air Execution is Vacuous").
where sdep is the set of semantic dependencies within
each thread and rf is the set of store-to-load ``reads-from'' links,
whether within a thread (rfi) or between threads (rfe).
% @@@ Consider adding citation defining rf, rfi, and rfe.

This is a fine definition and is consistent with the words in the C++
standard, but it has a problem with intrathread rfi links
as exemplified by the following code:
\begin{quote}
\scriptsize
\begin{verbatim}
 1   int r2 = atomic_load_explicit(&x, memory_order_relaxed);
 2   atomic_store_explicit(&y, r2, memory_order_relaxed);
 3   int r3 = atomic_load_explicit(&y, memory_order_relaxed);
 4   atomic_store_explicit(&z, r3, memory_order_relaxed);
\end{verbatim}
\end{quote}
This is an elaboration of \co{thread2()} from
Listing~\ref{lst:Simple OOTA}
that adds \co{z} along with lines~3 and~4.
The problem is that a C++ implementation may
note that line~3 could well execute immediately after line~2, giving
other threads no chance to modify \co{y} in between.
Such an implementation might therefore behave as if the source code
had instead been as follows:
\begin{quote}
\scriptsize
\begin{verbatim}
 1   int r2 = atomic_load_explicit(&x, memory_order_relaxed);
 2   atomic_store_explicit(&y, r2, memory_order_relaxed);
 3   // int r3 = atomic_load_explicit(&y, memory_order_relaxed);
 4   atomic_store_explicit(&z, r2, memory_order_relaxed);
\end{verbatim}
\end{quote}
Here line~3 has been optimized away in favor of line~4 storing the same value
to \co{z} that was stored to \co{y} by line~2.
And given that the load from \co{y} no longer exists, it cannot possibly
act as a temporal constraint.

In order to avoid these rfi links we will substitute rfe
for rf, defining an OOTA cycle---for now---as a cycle in sdep $\cup$ rfe.
Any rfi links in a cycle can instead be interpreted as part of sdep.
Although this does shunt additional complexity onto the term
``semantic dependency'', it also enables us to cleanly separate
the interthread and intrathread portions of any given OOTA cycle.
% \footnote{
%	Thanks to Alan Stern for providing a litmus test that demonstrated
%	that we had unwittingly (but productively!) shifted our definition
%	to sdep $\cup$ rfe.}

% We no longer need this.  I will move A.1 to the cbmc appendix.
% Please see
% Appendix~\ref{app:Informal Definition of Semantic Dependency}
% for more a more detailed (albeit quite informal) definition.
% Maybe eventually have an appendix that lists all the definitions
% of both ``OOTA cycle'' and ``semantic dependency''.

The inability of rfi links to act as temporal constraints is not the
only, or even the main, weakness in the intuitive argument against
OOTA cycles.
The primary difficulty lies in the fact that the code transformations
performed by optimizing compilers can destroy dependencies, including
semantic ones (depending on one's definition).
That is, even when the potential for a dependency exists in the
source code for a thread, there might be no dependency in the machine
code produced by a compiler.
There would then be no constraint forcing the implementation to execute
the thread's store later in global time than the load it supposedly depends on,
and thus no impediment to the occurrence of an OOTA cycle.
We will see examples of this destruction in
Sections~\ref{sec:Global Optimization Can Destroy Dependencies}
and~\ref{sec:Inventing Atomic Loads Can Destroy Semantic Dependencies} below.

\subsection{Properties of Semantic Dependencies}
\label{sec:Properties of Semantic Dependencies}

This section uncovers some semantic-dependency complexities.
Section~\ref{sec:Semantic Dependencies and Source Code}
shows that semantic dependencies are functions of executions,
rather than being strict functions of the source code.
Section~\ref{sec:Semantic Dependencies Can Be Many-To-One}
shows that semantic dependencies do not necessarily extend
from a single load to a single store, but can instead involve
multiple loads.
Section~\ref{sec:Semantic Dependencies Affected by Cross-Thread Optimizations}
shows that semantic dependencies can be affected by cross-thread
optimizations.
Sections~\ref{sec:Semantic Dependencies Affected by if Statements}
and~\ref{sec:Semantic Dependencies Not Affected by if Statements}
show that \co{if} statements can have surprising effects on semantic
dependencies, up to and including eliminating them completely.
Finally,
Section~\ref{sec:Semantic Dependencies and Matching Up Stores}
demonstrates some challenges in determining which of a group of
stores is involved in a given semantic dependency.

\subsubsection{Semantic Dependencies and Source Code}
\label{sec:Semantic Dependencies and Source Code}

Some discussions of semantic dependencies assume that they
are strictly functions of the source code.
Although there are ways of making this work, many instances of
semantic dependency must be considered functions of particular executions.
Consider for example:
\begin{quote}
\begin{verbatim}
x = y * z;
\end{verbatim}
\end{quote}
(Here and below, we have written shared-variable accesses without
annotations, for brevity.  Please imagine they are all relaxed atomic.)

As long as \co{z} is zero, changes in the
value of \co{y} will not cause a change in the value stored to \co{x}.
As a result, the semantic dependency from \co{y} to \co{x} exists only
in executions where \co{z} is nonzero,
which shows it is a property of the execution, not just of the source code.

\subsubsection{Semantic Dependencies Can Be Many-To-One}
\label{sec:Semantic Dependencies Can Be Many-To-One}

Suppose that in some execution of the previous example,
both \co{y} and \co{z} are zero.
Then changes to either \co{y} or \co{z}
will not cause a change in the value stored to \co{x}.
In other words, in this execution there is no semantic dependency
from either \co{y} or \co{z} to \co{x}.
But there \emph{is} a semantic dependency from the \emph{pair}
\{\co{y},~\co{z}\} to \co{x},
because changes to both \co{y} and \co{z}
can cause the value stored in \co{x} to change.
This means that, prior work~\cite{PaulEMcKenney2016OOTA}
notwithstanding, accurate definitions of sdep cannot always rely on
single-variable perturbations;
they must consider changes to multiple variables.
See Appendix~\ref{app:Non-Trivial Semantic Dependencies}
for examples and additional discussion.

Since we can no longer regard sdep as always relating a single load to a store,
the notion of a cycle involving sdep appears problematic.
We are forced to change our definition of an OOTA cycle again;
we will say that an execution is an instance of OOTA if in that execution:
\begin{quote}
	There are stores $W_0,$ \ldots,~$W_m$,
	where each $W_i$ semantically depends on a set of loads
	$\{R_{i,0},$ \ldots,~$R_{i,n_i}\}$,
	such that each $R_{i,j}$ reads from one of the $W_k$
	stores in a different thread.
\end{quote}
This makes OOTA more complicated than a simple cycle but
we will continue to refer to ``OOTA cycles'' out of habit.
Note that this new definition includes and generalizes the earlier
``cycle in sdep $\cup$ rfe'' definition.

\subsubsection{Semantic Dependencies Affected by Cross-Thread Optimizations}
\label{sec:Semantic Dependencies Affected by Cross-Thread Optimizations}

Consider the following:
\begin{quote}
\begin{verbatim}
x = y - z;
\end{verbatim}
\end{quote}
There appear to be semantic dependencies from \co{y} to \co{x} and from \co{z}
to \co{x}.
However, if the implementation somehow knows that \co{y} is
always equal to \co{z} at this point then there is no semantic dependency;
the implementation can act as if the statement were simply ``\co{x = 0}''.
We leave aside the question of how the implementation would know this,
given that \co{y} and \co{z} cannot be updated simultaneously\footnote{
	At least not by any means within the confines of the standard.}
and are subject to change at any time by
other threads (a point we will return to in
Section~\ref{sec:Global Optimization Can Destroy Dependencies}).
% The usual way in the Linux kernel is via a union, but it is probably
% best to ignore that possibility.

\subsubsection{Semantic Dependencies Affected by \co{if} Statements}
\label{sec:Semantic Dependencies Affected by if Statements}

Consider the following \co{if} statement:
\begin{quote}
\begin{verbatim}
r1 = x;
if (r1 > 0)
    y = r1;
else
    z = r1;
\end{verbatim}
\end{quote}
Here there is a semantic dependency from \co{x}, but in some executions
it extends to \co{y} and in others to \co{z}.
This is an example of a load affecting not the value of
a given store, but rather whether or not that store is executed at all.

\subsubsection{Semantic Dependencies Not Affected by \co{if} Statements}
\label{sec:Semantic Dependencies Not Affected by if Statements}

Compare this example to the previous one:
\begin{quote}
\begin{verbatim}
if (x > 0)
    y = 42;
else
    y = 42;
\end{verbatim}
\end{quote}
Because the stores executed on each arm of the \co{if} statement write
identical values to identical addresses, one could equally well regard
the two statements as performing two different stores or as performing
for all intents and purposes a single store, independent of \co{x}.
Reasonable C++ implementations might disagree on this matter and
therefore on whether or not the example has a semantic dependency.
It is the implementation's choice.

\subsubsection{Semantic Dependencies and Matching Up Stores}
\label{sec:Semantic Dependencies and Matching Up Stores}

Suppose we take the view that the previous example involves only one
store.
This opens up the door to greater complexity:
\begin{quote}
\begin{verbatim}
if (x > 0) {
    L1: y = 42;
} else {
    y = 53;
    y = 42;
}
\end{verbatim}
\end{quote}
Consider an execution in which \co{x} is greater than zero, so the
statement labeled \co{L1} runs.
Is it semantically dependent on \co{x}?
The answer isn't immediately clear.
If the other arm of the \co{if} is taken then a store of the same value 42 to
\co{y} occurs, but 53 is written before it.
Which of these two stores should be compared with the store in \co{L1}?

One way to cut the Gordian knot is to match up the stores by the order
they occur:
Since \co{L1} is the first store to \co{y} in its arm of the \co{if}
statement, it should be matched up with the first store to \co{y} in
the other arm.
Those two stores write different values so there is a semantic
dependency.

On the other hand, a compiler may decide to drop the \co{y =
53} store entirely, leaving it out of the machine code,
on the grounds that it's always possible for the two adjacent stores
to \co{y} to execute in such quick succession that no other thread
manages to read the value 53 before it gets overwritten with 42.
If the compiler does this then the first store to \co{y} in that
arm of the \co{if} statement \emph{would} agree with the store in
\co{L1}, and so there would not be a semantic dependency.
Once again, the decision is up to the implementation.

\medskip

We have seen several examples showing that semantic dependencies may vary
according to the execution and even the implementation.
This raises several questions, of which the first is:
What exactly is an execution?

\section{What is an Execution?}
\label{sec:What is an Execution?}

This section looks more carefully at executions, in some cases revisiting
example code from
Section~\ref{sec:Properties of Semantic Dependencies}.
Section~\ref{sec:Abstract Executions}
looks at executions from the viewpoint of the abstract machine, and
Section~\ref{sec:Hardware Executions}
looks at them from the viewpoint of the computer hardware.
Finally,
Section~\ref{sec:Relation Between Abstract and Hardware Executions}
describes how to relate these two viewpoints.

\subsection{Abstract Executions}
\label{sec:Abstract Executions}

The C++ standard describes the execution of a program in terms of
``a parameterized nondeterministic abstract machine'' in 4.1.2p1
(\co{[intro.abstract]}).
This description specifies how the abstract machine carries out the
operations of a source program in great, but not complete, detail:
\begin{itemize}
\item	Some of the abstract machine's characteristics are
	implementation defined, including things like the number of
	bits in the various integer types
	or whether the \co{char} type is signed.
\item	Some aspects of an execution are unspecified or nondeterministic,
	including things like the order of evaluation of the operands
	of most binary operators or of the arguments in a function call.
	Implementations may choose from a set of allowed behaviors.
\item	Some actions are deemed to have undefined results; the standard
	says essentially nothing about programs that can give rise to
	undefined behavior.
\item	Asynchronous actions (i.e., signal handlers) are largely ignored.
\item	Input and output are not described in any detail.
\end{itemize}
In addition to these points, the standard does not specify which store
an atomic load must read from, beyond requiring that the overall
pattern of loads and stores be consistent with the C++ memory model.
We assume that programs will not indulge in any computations that
could vary spontaneously from one execution to another,
such as basing a dependency on the time of day or a process ID.

The implementation-defined aspects can affect whether or not an
abstract execution contains a semantic dependency.
As an example consider the following, where the type of \co{c} is \co{char}:
\begin{quote}
\begin{verbatim}
y = (c >= 0);
\end{verbatim}
\end{quote}
Here \co{y} is semantically dependent on \co{c} in executions
running on implementations in
which \co{char} is a signed type,
but not those for which it is unsigned.

The same is true for the nondeterministic aspects of an execution.
Consider this example, with \co{i} initially zero:
\begin{quote}
\begin{verbatim}
int foo(int a, int b)
{
   return a / b;
}

r1 = foo(++i, ++i);
x = r1 * z;
\end{verbatim}
\end{quote}
Because early C~implementers could not come to agreement, the standard
does not specify the order of evaluation of function arguments, so
the value calculated for \co{r1} might be zero ($1/2$ truncated) or two
($2/1$).
In the former case there is no semantic dependency from \co{z} to \co{x},
but in the latter case there is.\footnote{
	Thanks to Peter Sewell for pointing out this possibility.}

(According to the current version of the standard, conflicting side effects
in unsequenced subexpressions constitute undefined behavior,
although there are proposals to make them defined in both
C++~\cite{GabrielDosReis2016P0145r3}
and C~\cite{AlexCeleste2023N3203}.
Nevertheless, the example above is allowed because the order of evaluation
of arguments to a function call is ``indeterminately sequenced''
(7.6.1.3p7 \co{[expr.call]}) rather than unsequenced, a subtle distinction.)

\medskip

The abstract executions we use will be fully specified.
This means that all the missing information must be supplied:
the implementation-defined characteristics, the selections for the
nondeterministic pathways, and most notably, for each load, the store
from which it reads and the value of the load.
We ignore issues of signal handlers and I/O;
in any case our litmus-test programs don't use them
(but see the discussion of volatile loads in
Section~\ref{sec:Relation Between Abstract and Hardware Executions} below).
The totality of this information---along with the program's source
code, of course---determines within each thread a unique, linearly
ordered series of steps to be carried out by the abstract machine.
However, with a few exceptions\footnote{
	Such as a load-acquire synchronizing with a store-release.}
there is no ordering relation between steps carried out
in different threads.
Even if a relaxed atomic load in one thread reads from a relaxed
atomic store in another thread, the standard does not require the
store to come before the load in any meaningful way.

With the compiler-based implementations we are considering,
the choices for the nondeterministic pathways are ``frozen'' into the
machine-code executable file and thus are completely determined
at runtime.
A consequence of this is that if two abstract executions of the same
thread under the same implementation agree on the values obtained by
the load operations during their first $N$ steps then they will agree
in every respect during those steps, although they may diverge later.

\subsection{Hardware Executions}
\label{sec:Hardware Executions}

The outcome when a given computer executes the machine code in a file
has historically been much better defined than the executions of the
C++ abstract machine.
The hardware's behavior is typically specified with great precision by
the designer or manufacturer, and there are formal, executable memory models
describing exactly what patterns of loads and stores can occur.
Thus, leaving aside questions of asynchronous interrupts and system
calls, the behavior of a CPU executing a particular thread within a
program is entirely determined by the values obtained by the various
memory-load instructions.\footnote{
	We regard read-modify-write instructions as consisting of both a
	memory load and a memory store.}

For this reason, the hardware executions we use will comprise (along
with the machine code being run) the computer architecture and for each
load instruction, the store instruction from which it reads and the
value obtained.
At this level, the fact that the original program was in C++ is
irrelevant; the same concepts apply to the execution of a program in
any compiled language.

A computer may execute the instructions in a thread out of order.
The architecture specifies the extent to which this may happen, and it
also specifies circumstances under which some pairs of instructions
must be executed in order.
Nevertheless, we will consider an execution to be determined by the
values obtained by its loads.
As with abstract executions, if two hardware executions of the same
thread on the same type of computer agree on the values obtained by
the load instructions during their first $N$ steps then they will
agree in every respect during those steps, although they may diverge
later.

\subsection{Relation Between Abstract and Hardware Executions}
\label{sec:Relation Between Abstract and Hardware Executions}

The C++ standard requires that for any valid implementation, when a
program runs its observable behavior must be the same as that of some
abstract execution of the source code given the same input (in the
absence of any abstract executions containing undefined behavior).\footnote{
	This requirement is the standard's ``as-if'' rule.}
This means:
\begin{itemize}
\item	The program's output must be the same as that of the abstract
	execution.
\item	Volatile accesses ``are evaluated strictly according to the
	rules of the abstract machine'' (4.1.2p6.1 \co{[intro.abstract]}).
\item	(There is a condition on the timing and interleaving of input and output,
	which does not matter for our purposes.)
\end{itemize}
We will say that the abstract execution is \emph{realized by} the
hardware execution.

Under any particular implementation,
a single program can have many different abstract executions,
varying in their decisions about which store each load reads from
and thus the value obtained.
It's worth noting, however, that not all the possible abstract executions
of a program need be realizable by the machine-code executable file
produced by that implementation.
In fact, we will see that \emph{none} of the possible OOTA executions
allowed by the loose C++ abstract machine will ever be realized
by the executables produced by many compilers,

Exactly what the standard's restriction on volatile accesses means
isn't entirely clear.
The handling of volatiles, as understood by compiler developers, has
been described as more folklore or a gentlemen's agreement than
anything else.
To help guide C++ users and implementers, the standard adds these
suggestive comments (9.2.9.2p5 and~6 \co{[dcl.type.cv]}):
\begin{quote}
	The semantics of an access through a volatile glvalue are
	implementation-defined.

	\co{volatile} is a hint to the implementation to avoid aggressive
	optimization involving the object because the value of the object
	might be changed by means undetectable by an implementation.
\end{quote}

Taking our cue from the folklore, we propose to recognize formally
that programs with volatile objects can execute in two different kinds
of environment: a benign one in which accesses to these objects work
the same as nonvolatile memory accesses, and a nonbenign one in which
accesses to volatile objects are subject to outside interference and
act more like I/O.
In particular, when it runs in a nonbenign environment, a program's
volatile loads can return unpredictable values.
They don't necessarily read from stores (in contrast to nonvolatile loads,
which always must return the value of the store they read from).
This implies that volatile load-acquires do not synchronize with
volatile store-releases in the sense of the C++ memory model,\footnote{
	However, they might instead synchronize with store-releases in
	device firmware (or vice versa), roughly speaking.}
so they do not contribute to the happens-before relation.
Also, in these environments the rfe relation does not apply to volatile
loads and stores, and hence the accesses in an OOTA cycle must be nonvolatile.

Of course, the machine-code file produced by a compiler must work
properly in either kind of environment.
Therefore the compiler must generally treat accesses to volatile objects
as a form of I/O, and it may not
invent, omit, merge, or reorder these accesses, as we will discuss in
Section~\ref{sec:Volatile and Quasi Volatile Accesses} below.

Given this relation between abstract and hardware executions, it is time
to turn our attention to the tools that manage hardware executions so as
to enforce that relation, namely, compilers.

\section{C++ Compilers}
\label{sec:C++ Compilers}

A complete C++ implementation consists of much more than just a compiler.
Among other things, for example, it might have a collection of
\co{.h} header files, a linker, runtime libraries, and a dynamic loader.
Nevertheless, for our purposes the compiler is the most important
component because it is what primarily determines the translation from
a C++ source program to directly executable machine code.
We will therefore use the terms ``compiler'' and ``implementation''
interchangeably.

Section~\ref{sec:Users Influence the Behavior of Compilers}
shows that C++ compilers can be influenced by their users as well
as by the standard.
Section~\ref{sec:Global Optimization Can Destroy Dependencies}
presents an example showing that global optimizations can destroy
semantic dependencies, and then
Section~\ref{sec:Inventing Atomic Loads Can Destroy Semantic Dependencies}
analyzes examples showing that inventing atomic loads can destroy
semantic dependencies.
Section~\ref{sec:Volatile and Quasi Volatile Accesses}
then presents required properties of volatile atomic operations, and
also defines properties of quasi-volatile atomic operations.

\subsection{Users Influence the Behavior of Compilers}
\label{sec:Users Influence the Behavior of Compilers}

The exact definition of a computer language is a subject of some debate,
with standards, implementations, and users all having some degree of
influence~\cite{KayvanMemarian2016DepthOfC-1,KayvanMemarian2016DepthOfC-2},
and each being prone to change over time.
In areas that are not well settled or where users might reasonably
want to resist the dictates of the standard,
compilers often provide switches to override their default behaviors.
An example is GCC's \co{-funsigned-char} command-line argument,
which causes it to treat variables of type \co{char} as unsigned.
Many more examples of user control over language semantics
may be found through use of the command \co{make V=1}
in a Linux-kernel source tree and by the discussion
in~\cite{KayvanMemarian2016DepthOfC-1,KayvanMemarian2016DepthOfC-2}.

We will consider these user-specified compiler switch settings to fall
within the implementation-defined parameters of the C++ abstract machine.
They should be provided, implicitly or explicitly, as part of any
abstract execution.

\subsection{Global Optimization Can Destroy Dependencies}
\label{sec:Global Optimization Can Destroy Dependencies}

Recall the Simple OOTA example in Listing~\ref{lst:Simple OOTA} on
page~\pageref{lst:Simple OOTA},
in which \co{thread1()} loads the value of \co{x} and stores it in
\co{y} while \co{thread2()} does the reverse.
A globally optimizing loose C++ compiler given that program might
transform it to the following before translating it into machine code,
if the compiler is sufficiently perverse:
\begin{quote}
\begin{verbatim}
 1 atomic<int> x(0);
 2 atomic<int> y(0);
 3
 4 void thread1()
 5 {
 6   int r1 = 42;
 7   y.store(r1, memory_order_relaxed);
 8 }
 9
10 void thread2()
11 {
12   int r2 = 42;
13   x.store(r2, memory_order_relaxed);
14 }
\end{verbatim}
\end{quote}
The loads previously on lines~6 and~12 have been replaced by constants.
Such a transformation complies with the loose C++ standard,
even though the resulting executable file would produce an unintuitive
OOTA outcome every time it runs!

The only justification a compiler could have for generating output like
this is that it knows exactly what accesses will be performed by both
threads, and therefore it knows that it will not violate the loose C++
memory model by assuming each thread's load reads from the other's
store.\footnote{
	A less perverse compiler could choose to avoid the
	OOTA cycle simply by not making this transformation.}
A similar justification can underlie the reasoning in
Section~\ref{sec:Semantic Dependencies Affected by Cross-Thread Optimizations};
in principle an analysis of the complete program could lead a compiler
to conclude that \co{y} will always be equal to \co{z} whenever a
particular \co{y - z} expression is evaluated, allowing the compiler
to replace the expression with a constant \co{0}.

By contrast, a compiler that analyzes only one thread at a time when
performing its optimizations and other code transformations will not
have this kind of global knowledge, and consequently it would not
perform the OOTA-ful transformation shown here.

Because we seek to find characteristics of compilers that will
guarantee the absence of OOTA behavior in the machine code they generate,
we will for now confine our attention to compilers that analyze only
one thread of source code at a time.
In more precise terms, we want the compilers under consideration
always to generate the same machine-code output for threads having
the same source code, regardless of the rest of the code in the programs
containing those threads.
Later on we will return to globally optimizing compilers.

\subsection{Inventing Atomic Loads Can Destroy Semantic Dependencies}
\label{sec:Inventing Atomic Loads Can Destroy Semantic Dependencies}

Consider this code, in which the final values of \co{r1} and~\co{r2}
are observable:
\begin{quote}
\begin{verbatim}
   int r1 = (x != 0);
   int r2 = (y != 0);
   z = (r1 == r2);
\end{verbatim}
\end{quote}
It is clear that the store to \co{z} semantically depends on the load
from \co{y}, because the value of \co{z} will change whenever \co{y}
changes between zero and nonzero (all else being equal).

However, an especially devious compiler might transform the source into
the following form before translating it to machine code:
\begin{quote}
\begin{verbatim}
 1  int r1;
 2  int r1a = (x != 0);
 3  int r1b = (x != 0);
 4  int r2 = (y != 0);
 5  if (r1a != r1b) {
 6      r1 = r2;
 7      z = 1;
 8  } else {
 9      r1 = r1b;
10      z = (r1 == r2);
11  }
\end{verbatim}
\end{quote}
The idea is that \co{r1a}, \co{r1b}, and \co{r2} can each be only zero or one,
so if \co{r1a} and \co{r1b} differ then one of them must be equal to \co{r2}.
In executions where this happens---because another thread writes to \co{x}
between the two loads---the implementation can choose at runtime to use
for \co{r1} whichever value agrees with \co{r2}, as shown on line~6.
Then the value stored to \co{z} on line~7 will always be one,
with no dependence on the value loaded from \co{y}.

A noteworthy aspect of this transformation is that it invents an atomic load:
The original form of the code reads \co{x} only once,
whereas the transformed code reads it twice.
Therefore we can rule out transformations like this one by insisting
the compiler not invent (or duplicate) atomic loads.\footnote{
	This issue is discussed more fully in
	Appendix~\ref{app:Inventing Atomic Loads},
	which includes other examples of invented atomic
	loads leading to incorrect behavior, that is, behavior not
	exhibited by the abstract machine.
	See also the ``Invented Atomic Loads'' paragraph of
        Section~\ref{sec:Constraints of the Standard}.}
In fact, we will require that atomic accesses be treated as
``quasi volatile'', in that the compiler is allowed to omit,
merge, or reorder them but not invent them.

Just what does this mean?

\subsection{Volatile and Quasi Volatile Accesses}
\label{sec:Volatile and Quasi Volatile Accesses}

Declaring objects to be volatile is a way for the programmer to
indicate that the hardware should perform all accesses to these
objects exactly as written in the source code, perhaps because they
represent memory-mapped device registers or DMA buffers rather than
normal memory locations.
In any event, we expect compilers' translations of volatile-object accesses
into machine code to be as close to verbatim as possible.

To express this idea in more formal terms,
and to explain what we mean by ``quasi-volatile'' object accesses,
we augment the
requirements for a hardware execution $H$ to realize an abstract
execution $A$.
Each realization must include a map from the set of accesses of
volatile objects in $A$ to the set of instructions in $H$ that access
these objects, having the following properties:
\begin{itemize}
\item	The map connects accesses of the same type (loads to loads
	and stores to stores) and to the same object.
\item	The map connects accesses in a thread of $A$ to accesses in
	the corresponding thread of $H$.
\item	The map is value-preserving: The value of an access in $A$ must be
	the same as the value of the access it maps to in $H$.
\item	In benign environments the map must preserve the rf relation.
	That is, if volatile load $R$ in $A$ reads from store $W$ then
	then the instruction it maps to in $H$ must read from the
	instruction that $W$ maps to.
\item	The map is order-preserving: Two accesses in the same thread
	of $A$ must map to accesses occurring in the same order in $H$.
	(In other words, the compiler may not reorder accesses
	to volatile objects.)
\item	The map is onto: For every access in $H$ to a volatile object
	there must be an access in $A$ that maps to it.
	(In other words, the compiler may not invent accesses to
	volatile objects.)
\item	The map is one-to-one: Different accesses in $A$ map to
	different accesses in $H$.
	(In other words, the compiler may not merge accesses to
	volatile objects.)
\item	The map is total: Every access in $A$ to a volatile object
	maps to some access in $H$.
	(In other words, the compiler may not omit accesses to
	volatile objects.)
\end{itemize}
Most of these are direct consequences of the fact that volatile-object
accesses are considered to be a form of I/O when the program runs
in a nonbenign environment.
But to be clear, these requirements apply in all environments.

By contrast, accesses to quasi-volatile objects are normal memory
accesses, not subject to unpredictable interference in nonbenign
environments (otherwise the program's behavior would be undefined).
However, we do impose most of the requirements above on quasi-volatile
object accesses.
The last two bullet points are left out: Compilers are allowed to merge or
omit accesses to these objects.
Because of this, the bullet point about preserving the rf relation
applies only when $R$ is not omitted, in which case $W$ must not
be omitted either, but now it applies in all environments.
Lastly, the requirement for order preservation is weakened; it applies
only to pairs of accesses to the same quasi-volatile object.
Accesses to different objects may be reordered relative to each other.

Section~\ref{sec:Constraints of the Standard}
presents examples illustrating some of these requirements.

\section{Hardware Dependencies, Instruction Ordering, and the
Fundamental Property}
\label{sec:Hardware Dependencies, Instruction Ordering, and the
Fundamental Property}

Section~\ref{sec:Dependencies at the Hardware Level}
discusses hardware-level dependencies involving machine instructions.
Section~\ref{sec:Instruction Ordering}
then examines the relationships between various instruction-ordering
mechanisms and semantic dependencies.
This material feeds into
Section~\ref{sec:The Fundamental Property of Semantic Dependencies},
which presents the fundamental property of semantic dependencies
and related complications.

\subsection{Dependencies at the Hardware Level}
\label{sec:Dependencies at the Hardware Level}

Dependencies between machine instructions can be understood in terms
of the flow of information within a CPU.
Each instruction has a set of inputs and is the source of a set of outputs,
some of which flow to the inputs of later instructions.
The inputs determine what an instruction will do.
A few examples:
\begin{itemize}
\item	An \co{add} instruction would typically have two inputs
	(the register values to be added together) and two outputs
	(the sum to be stored in a general-purpose register and some
	condition-code bits---e.g., Zero, Carry, and Overflow---to be
	stored in a status register).
\item	A conditional-move instruction would have as inputs the
	condition-code bits to test, the source register value, and the
	target register value; the output would be the value to be stored
	in the target register.
\item	A conditional- or computed-branch instruction would have as inputs
	the condition-code bits to test, the address of the following
	instruction (to be used if the condition is false) and the
	destination address (to be used if the condition is true).
	The output is the new address to be written to the
	instruction-pointer register.
\item	A memory-load instruction's input is the address to load from,
	and its output is the value obtained by the load, to be stored
	in a register.
\item	A memory-store instruction's inputs are the value to store and
	the address at which to store it; there are no outputs.
\end{itemize}
Note: Hardware dependency analysis considers only the information
flowing \emph{within} a CPU, not information flowing between the
CPU and memory, which is handled separately as part of the action
taken by an instruction.

Using this scheme, we say that an instruction $J$ is dependent on
another instruction $I$ when any of $I$'s outputs flow into $J$'s inputs,
perhaps indirectly via some intermediate instructions.
Tracing back the flow of information between instructions, you can
see that any hardware dependency must ultimately emanate from a load
instruction or the thread's initial register values (or possibly some sort of
input instruction, but we will not consider that complication here).
These are the only sources of truly new information in a thread.

The concept is simple, but it is important because of the way
dependencies affect instruction ordering in weakly ordered architectures.

\subsection{Instruction Ordering}
\label{sec:Instruction Ordering}

A CPU may start executing an instruction speculatively, but at some
point it must \emph{commit} to a decision either to definitely execute the
instruction's action with some collection of well defined inputs and outputs
or else to abandon it.
Thanks to the law of cause and~effect, a CPU is not able to commit an
instruction or its outputs until the relevant inputs' sources have committed.
The reason is obvious: An input is subject to change at any time until
its source commits to its value,
and the instruction and its outputs can't commit until the CPU has fully
determined what it will do and what they will be.
(On the other hand, the CPU need not wait for inputs that won't affect
the instruction's results.
For instance, a conditional-move instruction wouldn't need to wait for
the source register input to be determined once it knew that the
condition was definitely false.
And, although no architectures we are aware of do this, there is nothing
in principle to prevent a CPU from committing a \co{multiply}
instruction as soon as either one of its inputs' sources has committed to the
value zero.)

Of course, an instruction's inputs don't all have to be outputs from
earlier instructions; some of them may be immediate constants present
in the instruction itself.
In this case there is no need to wait for those inputs because
their values are fully determined from the start.

The overall effect of these hardware dependencies is that if a change to an
output of instruction $I$ would lead to a change in the action or outputs of
a later instruction $J$, then the CPU must commit $I$'s output before
committing $J$'s action and outputs.
Thus dependencies force instructions to commit in order, even on
weakly ordered architectures.

We will say that one instruction is \emph{ordered after} another to mean that
the CPU forces it to commit after the other one commits.
Hence instructions are ordered after the instructions they depend on.
(Note that this implies nothing about when a load instruction retrieves
its value from memory; it may do so long before it or a prior
instruction commits.)

Dependencies aren't the only mechanism that can lead to
ordering in hardware executions.
The simplest alternative is a load reading from a store in another
thread, i.e., the rfe relation.
As mentioned in Section~\ref{sec:Simple OOTA Cycle}, a CPU does
not make the value of a store instruction available for other CPUs to
load until the store commits or some time later.
And after this happens, it takes some time for the information about
the store to travel from that CPU to others, owing to the inescapable
facts that processors have nonzero size and information cannot travel
faster than the speed of light.
Since the load instruction cannot commit until its output (the value it
reads) is fully determined, it must commit after the store it reads
from.
(To be fair, this should be considered more as a \emph{result} of
ordering than as a \emph{cause} of ordering, in that
if a load commits before a store on another thread then it
cannot possibly read from that store.)

A similarly straightforward mechanism for ordering is conditional or
computed branches.
An instruction executing after such a branch cannot commit
until the CPU has committed to whether the branch will be taken and
if taken, where it will branch to (that is, the output value it will
send to the instruction-pointer register);
until then the CPU cannot know whether the later instruction should
be executed at all.
Therefore instructions following a conditional- or computed-branch
instruction in a hardware execution must commit after the branch, and hence
after the source for the branch's condition or destination input.
(There could theoretically be an exception for a branch that
conditionally jumps to the immediately following instruction.
We can ignore this possibility by treating such branches as no-ops.)

\subsection{The Fundamental Property of Semantic Dependencies}
\label{sec:The Fundamental Property of Semantic Dependencies}

With the contents of the last few sections under our belts, we can
formulate the Fundamental Property that we would like all semantic
dependencies to satisfy.
Note that this formulation makes sense only for implementations in
which all atomic objects are treated as volatile or quasi volatile.
\begin{quote}
Let $W$ be a store which semantically depends on loads
$\{R_0$, \ldots,~$R_n\}$ in some abstract execution $A$,
and suppose that $W$ is not
omitted in some hardware execution $H$ realizing $A$.
Then for some $i$, load $R_i$ is not omitted in $H$ and the
instruction it maps to is ordered before the instruction $W$ maps to.
\end{quote}
It's quite straightforward to show that under any implementation in which
semantic dependencies satisfy the Fundamental Property, no OOTA cycle
has a nontrivial realization.

Indeed, suppose that abstract execution $A$ is realized by hardware
execution $H$ and $A$ has an OOTA cycle.
This means there are atomic stores $W_i$ in $A$,
semantically depending on atomic loads
$\{R_{i,j}\}$ where each of the loads reads from one of the $W_k$
stores in a different thread.
Let $W_i'$ and $R_{i,j}'$ be the hardware instructions these accesses
map to in $H$, if they aren't omitted.
Assuming that the stores are not all omitted,
one of the $W_i'$ instructions, let's say $W_0'$, must commit first.
By the Fundamental Property, one of the loads that $W_0$ depends on,
let's say $R_{0,0}$, is not omitted and $R_{0,0}'$ is ordered before $W_0'$.
But now we have a contradiction:
\begin{itemize}
\item	$W_0'$ commits after $R_{0,0}'$;
\item	$R_{0,0}'$ commits after the store instruction $W_k'$ it reads from;
\item	$W_k'$ commits no earlier than $W_0'$.
\end{itemize}

If all the stores in the OOTA cycle are omitted then all the reads
must be omitted as well.
In effect, the entire cycle has been optimized
out of existence by the compiler.
Although we are unable to prove it, we conjecture that in this
situation there must be another abstract execution which has the
same observable effects as $A$ and is also realized by $H$, but in
which the OOTA cycle does not occur.
Thus there would be no way to tell, merely by observing the effects of
$H$, whether there was an OOTA cycle or not.
For this reason we declare realizations of OOTA cycles in which
all the accesses are omitted to be \emph{trivial}.

\section{A Definition of Semantic Dependency}
\label{sec:A Definition of Semantic Dependency}

Semantic dependency is a notoriously difficult concept to define
rigorously and precisely.
A large part of the reason is because it was never a completely clear
concept to begin with, especially when there are multiple accesses to
the variables involved.
In this section we will stick our necks out by offering just such a
definition.
No doubt many people will object to it for various reasons, but we
nevertheless hope it will help move the discussion forward.

The definition given below is applicable only to C++ implementations
that treat all atomic objects as though they are volatile or
quasi volatile.
(For compilers that perform only single-thread analysis, not global
analysis, quasi volatile is sufficient.)
In this setting we can relate abstract and hardware executions by
means of the map of accesses described in
Section~\ref{sec:Volatile and Quasi Volatile Accesses}.
The key insight is that this allows us to consider semantic
dependencies at the level of the machine code, where they are much
more tractable.

Section~\ref{sec:For Compilers Using Single-Thread Analysis}
focuses on compilers that restrict their analysis to a single
thread,
Section~\ref{sec:For Compilers Using Global Analysis}
relaxes this single-thread restriction,
Section~\ref{sec:Verifying the Fundamental Property}
verifies the fundamental property of semantic dependencies,
and
Section~\ref{sec:Outstanding Issues}
discusses general questions regarding our definitions.

\subsection{For Compilers Using Single-Thread Analysis}
\label{sec:For Compilers Using Single-Thread Analysis}

In this section we consider implementations whose compilers perform
only single-thread analysis and treat atomic objects as
quasi volatile.
This implies that if two different programs contain the same thread
(i.e., the same source code for the functions and objects in the thread), the
machine code generated by the compiler for the thread will be the
same in the two programs.

We begin by recognizing that semantic dependencies are relative to a
particular execution and a particular implementation,
as described in Section~\ref{sec:Properties of Semantic Dependencies}.
The same source code may or may not contain a semantic dependency,
according to the details of the execution in question and the
machine code produced by the compiler.
For this reason we will characterize semantic dependencies in a given
abstract execution realized by a given hardware execution.
(While it possible to argue about semantic dependencies in abstract
executions that have no hardware realizations, doing so seems pointless.)

Let $A$ be an abstract execution of some program $P$ containing a
thread $T$, and let $H$ be a hardware execution realizing $A$.
Let $W$ in $T$ be a store to an atomic object, and let $\{R_0,$
\ldots,~$R_n\}$ in $T$ be a set of loads from atomic objects on which
$W$ might or might not depend.
We can dispose of one case immediately: If $W$ is omitted in $H$ then
the issue of semantic dependency is moot.
You can give either answer since it will have no effect.
Therefore we'll assume that $W$ is not omitted.
Then:
\begin{quote}
There is a semantic dependency from $\{R_i\}$ to $W$ in $A$ and $H$,
relative to the compiler used to produce $H$, if there is another
abstract execution $B$ realized by hardware execution $G$ under the
same compiler that together \emph{witness the semantic dependency}.
\end{quote}

To be a proper witness, $B$ must be an execution of some program $Q$,
not necessarily the same as $P$ but which contains the same thread $T$.
The thread should start out with the same initial state in $A$ and
$B$, and all loads in $A$ coming before any of the $R_i$ should obtain
the same value as they do in $B$ (this is part of our interpretation
of ``all else being equal'').
It follows that the two abstract executions of $T$ will be identical
up to the first of the $R_i$ loads.

Let $W'$ and $\{R'_i\}$ be the accesses in $H$ that $W$ and the
non-omitted $\{R_i\}$ loads map to.
We then require that the hardware executions of $T$ in $H$ and $G$ be
identical for an initial period lasting up to the first of the $R'_i$.
Following this initial period there will be a common period, during
which $H$ and $G$ execute the same machine instructions but do not
necessarily compute the same values.
This common period ends when one of the hardware executions
takes a conditional branch that the other doesn't, or when a computed
branch leads to different addresses in the two executions, or when $T$
ends, whichever comes first.
Past this point $H$ and $G$ diverge and are no longer directly
comparable, as they execute different instructions.
Our third requirement for being a witness is that each load in the
common period either must obtain the same value in $H$ and $G$, or
must itself be one of the $R'_i$ loads, or must be ordered in $H$
after one of the $R'_i$ loads (this is the remaining part of our
interpretation of ``all else being equal''.)

Finally, we need to determine an instruction $X'$ in $G$ that
corresponds to $W'$.
If $W'$ is in the initial or common period of $H$ this is no
problem; we can take $X'$ to be $W'$ itself.
But if $W'$ is in the divergent part of $H$ then things aren't so
simple.
The choice is somewhat arbitrary, and so we will fall back on the
earlier proposal of matching up stores by the order they occur.
Let $y$ be the atomic object that $W'$ stores to, and suppose $W'$
is the $N$th store to $y$ within the divergent part of $H$.
Then $X'$ will be the $N$th store to $y$ in the divergent part
of $G$, if such a store exists.
Our last requirement for being a witness to a semantic dependency
is that $X'$ act differently from $W'$: it doesn't exist, it stores
a different value, or it stores to an object other than $y$.

\subsection{For Compilers Using Global Analysis}
\label{sec:For Compilers Using Global Analysis}

As promised earlier, we now consider implementations whose compilers
may use global analysis.
In order to obtain the desired results we have to require that these
compilers treat all atomic objects as volatile.
Equivalently, the machine code generated by such a compiler must be
the same for a given program as for a ``volatilized'' form of the
program in which all the atomic objects are defined to be volatile.

In this context our definition of semantic dependency is essentially
the same as before.
Since we can no longer expect the machine code for a thread to be the
same regardless of the program it belongs to, the program $Q$ in the
earlier definition (of which $B$ and $G$ are executions) must be
$P$ or its volatilized form.
However, we do now allow the possibility that the executions $B$ and
$G$ take place in a nonbenign environment.
Aside from these minor adjustments, the definition remains unchanged.

\subsection{Verifying the Fundamental Property}
\label{sec:Verifying the Fundamental Property}

Of course we want to check that our definition of semantic dependency
satisfies the Fundamental Property of
Section~\ref{sec:The Fundamental Property of Semantic Dependencies}.
Given the information we have already presented, the demonstration is
easy.

Suppose we have $W$, $\{R_i\}$, $A$, and $H$ as in the definition.
The Fundamental Property assumes that $W$ is not omitted in $H$, so
there is an abstract execution $B$ with hardware realization $G$
witnessing the semantic dependency of the store $W$ on the loads
$\{R_i\}$ in $A$ and $H$.
We must show that some $R_i$ is not omitted and $R'_i$ is ordered
before $W'$ in $H$.
The proof splits into three cases.

First case: $W'$ lies in the initial period of $H$.
During the initial period of the hardware executions, $H$ and $G$
behave identically and therefore $W'$ performs the same write in
both.
This contradicts the fact that $B$ and $G$ witness the semantic
dependency.

Second case: $W'$ lies in the common period of $H$.
Since the action of $W'$ in $H$ is different from its action in $G$,
at least one of its inputs must differ between the two hardware
executions.
Therefore the source instruction for that input must behave
differently, and so must one of its sources, going back until we reach
a load instruction that obtains differing values in $H$ and $G$.
Then $W'$ depends on this load and so is ordered after it.
And since the load must lie in the common period of $H$, by the
definition of semantic dependency it must either be one of the
$R'_i$ or be ordered after one of them.
Therefore $W'$ is ordered after one of the $R'_i$ in $H$,
which certainly means that $R_i$ is not omitted.

Third case: $W'$ lies in the divergent part of $H$.
This happens when $W'$ comes after the conditional or computed branch
which marks the end of the common period by going different ways in
$H$ and $G$.
Just as in the previous case, since the branch behaves differently in
the two executions it must be ordered after one of the $R'_i$ loads.
And then so must $W'$, because any instruction following a conditional-
or computed-branch instruction must commit after the branch commits.
QED.

\medskip

A corollary of this result is that if an implementation's compiler
either
\begin{itemize}
\item	uses single-thread analysis and treats atomic objects as
	quasi volatile, or
\item	uses global analysis and treats atomic objects as volatile,
\end{itemize}
then programs produced by that implementation will never exhibit OOTA.
Thus the implementation will automatically comply with full C++, even
though it may be been designed only to comply with loose C++.

\subsection{Outstanding Issues}
\label{sec:Outstanding Issues}

Here we consider some general questions related to our definition of
semantic dependency.

\subsubsection{Relative versus Absolute Dependency}
\label{sec:Relative versus Absolute Dependency}

A drawback of the definition is that it is only relative to a specific
compiler or implementation.
This may strike some people as wishy-washy and avoiding the real
problem, in that a given piece of code should either contain or not
contain a semantic dependency, independent of the implementation used to
run it.

We can address this drawback by defining an \emph{absolute semantic
dependency} as one that is present relative to any valid loose C++
implementation, past, present, or future, real or imagined.
Of course this notion has its own problems, including that it is
extremely nonconstructive and impossible to apply in practice.
However it may be the best we can do with our current understanding
of computing systems.

There is one thing we can definitely state:
Programs produced by an implementation of the right sort will never
exhibit absolute OOTA (that is, an OOTA cycle in which all the
semantic dependencies are absolute).
This is simply because an absolute semantic dependency is {\it
a fortiori\/} a semantic dependency relative to the compiler in use.

But in fact we have shown more than this:
Programs will never exhibit an OOTA cycle relative to the compiler
used to build them, even when that cycle is not absolute.
In this sense we have gone beyond the requirement of full C++.

\subsubsection{Global Analysis and Volatile versus Quasi Volatile}
\label{sec:Global Analysis and Volatile versus Quasi Volatile}

In principle we don't need to require global-analysis compilers to
treat atomic objects as volatile; our results would still hold if they
merely treated them as quasi volatile.
We chose not to do this because it would violate our intuitions about
semantic dependencies.

For example, consider the Simple OOTA program, repeated here in
simplified form:
\begin{quote}
\begin{verbatim}
void thread1() {
   int r1 = x;
   y = r1;
}

void thread2() {
   int r2 = y;
   x = r2;
}
\end{verbatim}
\end{quote}
A loose C++ compiler using global analysis and treating \co{x} and \co{y} as
quasi-volatile objects could omit the two loads, replacing them in the
machine code with simple assignments ``\co{r1 = 42}'' and
``\co{r2 = 42}''.
This would be a valid transformation, and the resulting behavior would
not count as OOTA according to our definition because the
dependencies in \co{thread1} and \co{thread2} would not be semantic.

To see why not, recall that a semantic dependency must have a witness,
another execution in which the store acts differently.
But this transformed program has no other hardware executions; every time it
runs it will store 42 to both \co{x} and \co{y}.
(Keep in mind also that since the atomic objects are not treated as
volatile, they are not subject to unspecified interference when the
program runs in a nonbenign environment.)

This unintuitive behavior could not occur if the two loads were not
omitted.
In fact, the definition of semantic dependency might remain
perfectly acceptable if the requirement for global-analysis compilers
were weakened, if the compiler treated atomic objects as quasi
volatile and in addition was not allowed to omit accesses to them.
This is a possible topic for future research.

\subsubsection{Effect of Memory Layout}
\label{sec:Effect of Memory Layout}

Part of our demonstration of the Fundamental Property of semantic
dependencies relies on the fact, stated in
Section~\ref{sec:Abstract Executions},
that an abstract execution of a thread is entirely determined by the
values obtained for its loads.
But when we compare abstract executions of the same thread in two
different programs, this may no longer be entirely true owing to the
effect of differing memory layouts.

Consider this simple example:
\begin{quote}
\begin{verbatim}
x = (int) &x;
\end{verbatim}
\end{quote}
Even though the example contains no loads at all, it may store
different values when running in different programs because the
object \co{x} may be allocated at differing addresses in those
programs.
According to our definition, this could count as a degenerate OOTA
cycle of length one, in which the store is semantically dependent on
an empty set of loads!

To rule out such pathological counterexamples we should require that
in a witness to a semantic dependency, the addresses of all the
objects and functions referred to in the thread $T$ are the same as
in the original execution.
This is a very technical restriction but there are occasions when
the issue might realistically arise, such as when computing a hash
value based on an object's address.

\subsubsection{Merging Quasi-Volatile Loads}
\label{sec:Merging Quasi-Volatile Loads}

The compiler is permitted to merge quasi-volatile loads.
This can lead to surprising results because a particular load
may be merged with an earlier load in one execution and with a
later load in another.
This is demonstrated in the following, which is a variant of
the example in
Section~\ref{sec:Inventing Atomic Loads Can Destroy Semantic Dependencies}:
\begin{quote}
\begin{verbatim}
int r1 = (x != 0);
int r2 = (x != 0);
int r3 = (x != 0);
int r4 = (y != 0);
z = (r2 == r4);
\end{verbatim}
\end{quote}
Consider an abstract execution in which \co{r1}, \co{r2}, and \co{r4}
are zero and \co{r3} is one (because another thread changed the value
of \co{x} between two of the loads).
We would expect that the store to \co{z} would be semantically
dependent on the load of \co{y} in this execution.

However, a single-thread analysis compiler can translate this into the
machine-code equivalent of:
\begin{quote}
\begin{verbatim}
int r1 = (x != 0);
int r2;
int r3 = (x != 0);
int r4 = (y != 0);
if (r1 != r3) {
   r2 = r4;
   z = 1;
} else {
   r2 = r1;
   z = (r2 == r4);
}
\end{verbatim}
\end{quote}
In effect, the \co{r2} load is merged with the \co{r1} load in
executions where \co{r1} is equal to \co{r3} or to \co{r4}, and it is
merged with the \co{r3} load in other executions.

To demonstrate the semantic dependency in the original code,
a suitable witness would have
to include a hardware realization of the abstract execution in which
\co{r1} and \co{r2} are zero and \co{r3} and \co{r4} are one.
But there are no hardware realizations of this execution with the
machine code indicated above!
Since \co{r1} is different from both \co{r3} and \co{r4}, the \co{r2}
load will be merged with the \co{r3} load and so \co{r2} will
necessarily be one, not zero.
Thus, relative to this compiler the example does not contain a
semantic dependency.

Although it is unexpected, we cannot say that this conclusion is
definitely wrong, because our intuitive notions of semantic
dependency are not clear in cases where multiple loads of the same
variable are present.
% Can we instead say that transforming the code might require that the
% witness also be transformed?

\medskip

Despite these outstanding issues, our definition of semantic dependency
allows us to bring to bear the real-world constraints outlined in the
next section.

\section{Real-World Constraints}
\label{sec:Real-World Constraints}

The C++ standard imposes many constraints, and these have been considered
in prior work.
In real-world C++ implementations, however, the standard's abstract machine
must be mapped onto real hardware, and this imposes additional
constraints stemming from hardware architecture and design.
This mapping and these constraints have been only partially accounted
for in the standard and other work.
Part of the reason for this is
that accurate and executable formal descriptions of hardware memory
models did not appear until after the standard was
released~\cite{JadeAlglave2011ppcmem,Maranget2012TutorialARMPower}.
But they do exist now.

The following sections discuss these constraints, starting with
hardware constraints and continuing with additional constraints imposed
by the standard.
An additional section discusses the OOTA-cycle implications for C++
tooling.

\subsection{Hardware Architecture and Design}
\label{sec:Hardware Architecture and Design}

Computer systems are expected to continue increasing their use of
speculative execution.
Nevertheless, the main points of this paper will remain unaffected.
To see why, keep in mind that on real systems observable effects
must eventually be committed, but no effect can be committed
while any of the computations that determine the effect remain
speculative.
Combining this with the obvious fact that a machine instruction cannot
commit until it is no longer speculative, we can draw two conclusions:
\begin{itemize}
\item	A load cannot commit until the store it reads from has
	committed.\footnote{
	An early ARMv8 memory model did allow loads that read from an
	earlier store in the same thread to commit before the store.
	This apparent exception merely reflects a difference in
	nomenclature; in the memory model a store was said to commit at
	the time when it made its new value available to other
	threads, whereas we say that a store commits at the time when
	the CPU has irrevocably decided that it will take place with a
	particular fully determined value and address.}
\item	A store cannot commit until all the loads on which it has a
	hardware dependency have committed.
\end{itemize}

As an example showing what can go wrong when these principles are
violated, consider Listing~\ref{lst:Speculated Store and Non-Speculated Load}.
Suppose that the storage for \co{x} is located immediately before that
for \co{y[]}, and
suppose further that hardware speculation incorrectly guesses the value
of \co{r1} will be $-1$, so that the speculated store of 42 on line~11 uses
the address of \co{x}.\footnote{
	Yes, \co{y[-1]} is UB, but the CPU neither knows
	nor cares, nor should it.}
Then if line~13 reads from the speculative store and commits, it will load
the value 42 into \co{r2}---an observable effect since \co{r2}
is mentioned in the \co{exists} clause on line~16---although
by the rules of the C++ abstract machine the value must instead be zero.

\begin{listing}[tbp]
@@ DisplayLitmus litmus/speculative-store.litmus @@
\caption{Speculated Store and Non-Speculated Load}
\label{lst:Speculated Store and Non-Speculated Load}
\end{listing}

Applied to Listing~\ref{lst:OOTA Cycle} on
page~\pageref{lst:OOTA Cycle}, for another example,
the principles dictate that if the load on line~8 reads from the store
on line~18 and the load on line~16 reads from the store on line~10
(as they do in the execution described by the listing's \co{exists} clause),
then each of the stores must commit before its corresponding load.
A further application says that
neither line~10 nor line~12 can commit until after the load
in line~8 (and also the conditional on line~9) has committed.
Similarly, neither line~18 nor line~20 can commit until after line~16
(and~17) has committed.
Taken together, these constraints imply that the OOTA cycle in the
listing cannot be realized, because to do so would require that
none of the machine instructions corresponding to lines~8, 10, 16,
and~18 could commit before the others.
Furthermore, it's not possible to circumvent this reasoning by suggesting
that some of those lines could commit at the same time,
because instructions take time to execute even when executing
speculatively.

Yes, this does mean that hardware can produce OOTA cycles during
speculative execution, but the hardware is required to prevent such
speculative cycles from committing.
As a special case of this requirement, if a load obtains
a value from a speculated store that has not yet committed,
and the store gets squelched,
then the load must be restarted before it commits even if the load and
the store are executed by different CPUs.

The requirement applies within a single
multithreaded core as well as between cores.
The fact that intracore communication is faster than intercore
communication does not magically give loads the ability to commit
before the stores they read from, no matter what cores they execute on.

\medskip

\begin{listing}[tbp]
@@ DisplayRunLitmus litmus/oota-causality-2.litmus @@
\caption{Causality Test Case 2}
\label{lst:Causality Test Case 2}
\end{listing}

Another feature we can expect of upcoming computer systems is the
addition of new hardware instructions.
For example, one could imagine a conditional-store instruction,
similar to existing conditional-move instructions but carrying out
a store to memory rather than a move to a register.
The condition would control whether or not the store takes place.
A compiler doing single-thread analysis could use a conditional-store
instruction to generate code corresponding to lines~10 and~11 of
Listing~\ref{lst:Causality Test Case 2}
on page~\pageref{lst:Causality Test Case 2}
without use of a conditional branch, potentially
enabling more hardware optimizations than would otherwise be permitted.
Nevertheless, the conditional store could not be committed until
the instructions it depends on (lines~8 and~9) have committed.
Any new instruction will be subject to this constraint,
just as the instructions in current systems are.

A key conclusion to draw from this discussion is that
regardless of how advanced a computer system may be,
it cannot commit a store having a hardware dependency on some
set of loads until after (in global time) those loads have committed.

\subsection{Constraints of the Standard}
\label{sec:Constraints of the Standard}

Volatile atomic accesses constitute observable
behavior~\cite[\co{intro.abstract}]{ThomasKoeppe2023N4950} and
must be executed in strict accordance with the rules of the C++ abstract
machine.
This means that a volatile atomic store in the source code
corresponds directly to a machine-code store instruction, which can
commit only after all loads whose values are used to compute the
store's address and value have committed.
OOTA aficionados will recognize this as a special case of ordering
relaxed loads before relaxed stores, albeit one not requiring
expensive memory-fence instructions on weakly ordered architectures.

Although C++ nonvolatile accesses to atomic objects are not observable
behavior, any compiler that restricts its code analysis to a
single thread must assume (unless it can prove otherwise) that a given
relaxed atomic store can affect observable behavior in a different
thread that loads the value stored.
This assumption does not constrain the compiler to anywhere near
the extent that a volatile relaxed atomic store would, but it does add
significant constraints over those related to nonvolatile non-atomic
stores.

For example, nonvolatile relaxed atomic accesses are subject to the
C++ memory model.
The ``order-preserving'' bullet point from
Section~\ref{sec:Volatile and Quasi Volatile Accesses}
prohibits reordering of quasi-volatile accesses when they are
to the same object,
but such reordering is in any case directly forbidden by the memory model
(the four coherence rules discussed in 6.9.2.2p14--19 \co{[intro.races]}).
The ``onto'' bullet point also largely reiterates restrictions that follow
from the standard, in particular,
that compilers should not invent atomic stores,
duplicate atomic stores, or invent atomic loads.
On the other hand, it is permissible for compilers to omit redundant
atomic stores or fuse nonvolatile atomic accesses of adjacent objects
under appropriate circumstances;
these topics correspond to the ``total'' and ``one-to-one'' bullet points of
Section~\ref{sec:Volatile and Quasi Volatile Accesses}.

\paragraph{Invented Atomic Stores}
\begin{listing}[tbp]
@@ DisplayRunLitmus litmus/invented-store.litmus @@
\caption{Example Invented Store}
\label{lst:Example Invented Store}
\end{listing}
The reason that atomic stores should not be invented is that doing so can
introduce new (and almost certainly undesirable) behaviors that are
forbidden by the abstract machine.
To see this, consider Listing~\ref{lst:Example Invented Store},
a \co{herd7} litmus test that demonstrates such a behavior.
Without line~8, only the values~0 and~3 can be loaded into \co{r1} on
line~13.
With line~8, the additional value~42 can also be loaded into \co{r1}.
The compiler is therefore forbidden from inventing the store of 42
unless it can prove that doing so does not negatively affect the
program's observable behaviors.
For example, the compiler might be able to prove that there
are no other accesses to \co{x} at the time of the invented
store---but it's very hard to imagine how a compiler that analyzes
only a single thread at a time could prove such a thing.

\paragraph{Duplicated Atomic Stores}
\begin{listing}[tbp]
@@ DisplayRunLitmus litmus/duplicated-store.litmus @@
\caption{Example Duplicated Store}
\label{lst:Example Duplicated Store}
\end{listing}
Duplicating atomic stores is a special case of inventing them, but it is
worth illustrating the fact that duplicating stores can also introduce
new and undesirable behaviors.
To see this, consider Listing~\ref{lst:Example Duplicated Store}, a
litmus test that demonstrates such a behavior.
Without line~9, if the value loaded into \co{r1} is one then the final value
of \co{x} must be two.
With line~9, the final value of \co{x} can be one even when the
value loaded into \co{r1} is one.
The compiler is therefore forbidden from duplicating atomic stores
unless it can prove that doing so does not negatively affect the
program's observable behaviors.
Again, the compiler might be able to prove that there
are no other accesses to \co{x} at the time of the duplicated store.

\paragraph{Invented Atomic Loads}
\begin{listing}[tbp]
\scriptsize
\begin{verbatim}
 1 struct foo {
 2   int a;
 3   int b;
 4 };
 5 _Atomic struct foo *globalfoop;
 6
 7 void bar()
 8 {
 9   int a;
10   int b;
11   struct foo *fp;
12
13   fp = atomic_load_explicit(&globalfoop, memory_order_acquire);
14   a = fp->a;
15   // fp = atomic_load_explicit(&globalfoop, memory_order_acquire);
16   b = fp->b;
17   do_something_with(a, b);
18 }
\end{verbatim}
\caption{Example Duplicated Load}
\label{lst:Example Duplicated Load}
\end{listing}
Although one can argue that it is functionally correct to invent atomic
loads as long as the loaded value is not used, doing so can result in
extra cache misses for no good purpose.
And duplicating an atomic load can also lead to incorrect results if done
carelessly.
To see this, consider
Listing~\ref{lst:Example Duplicated Load}.
Suppose that the load on line~13 were duplicated, for example, as if the
compiler had uncommented line~15.
Then the values of \co{a} and \co{b} passed to line~17's call to
\co{do_something_with()} might be from different instances of the
\co{foo} structure, something that \co{do_something_with()} probably
isn't prepared to deal with and that represents incorrect behavior,
that is, behavior that the abstract machine does not exhibit.
See
Appendix~\ref{app:Inventing Atomic Loads}
for more details on this example and other examples of invented
atomic loads.

For these reasons, C++ compilers should avoid inventing atomic
loads, and in fact should avoid even duplicating them.

\paragraph{Omitted Redundant Atomic Stores}
In contrast, a pair of back-to-back nonvolatile atomic stores to the
same object
always might be executed such that no other thread accesses the object
during the time between those two stores.
This means that if the compiler omitted the first store,
the user would have no way to prove this fact short of
inspecting the assembly code.
In cases where such omissions are undesirable, the user can resort to
volatile atomic stores or to inline assembly\footnote{
	For example, by placing the Linux-kernel \co{barrier()} macro
	between the two stores.
	This macro is an empty GCC \co{asm} that specifies the \co{memory}
	clobber.}
to prevent them.

Omitting a redundant store cannot create an OOTA cycle,
because any valid execution (OOTA or not)
of the program with the first of a back-to-back pair of nonvolatile
relaxed atomic stores omitted is a valid execution of the
program with the store present.
% This is awkward because a reader might reasonably extend the "Acting
% as if" all the way to the comma.  The best I could do was "It is not
% possible to create an OOTA cycle by acting as if a redundant store
% was omitted from the source code", which is admittedly not all that
% great.

\paragraph{Fused Accesses of Adjacent Nonvolatile Atomic Objects}
Suppose a pair of adjacent nonvolatile atomic objects, when combined,
form a single machine-word aligned and machine-word sized unit.
Suppose further that the source code contains a pair of atomic
loads, one from each of these objects, with the last of the pair
being a relaxed load.
Then the compiler could generate instead
a single atomic load of the combined pair,
with the memory-ordering semantics of the first load.
A similar fact applies to pairs of stores.

This holds even when the atomic objects are treated as quasi volatile,
providing an example of how the ``one-to-one'' bullet point of
Section~\ref{sec:Volatile and Quasi Volatile Accesses} does not apply
to them.

\subsection{Semantic Dependencies and Tooling}
\label{sec:Semantic Dependencies and Tooling}

Where a C++ implementation's primary function is to correctly execute
a C++ program, the function of tooling is often to evaluate properties
of possible executions of that same program.
While implementations should avoid breaking semantic dependencies,
tools are in some cases permitted to misclassify them, as shown in
Table~\ref{tab:Dependency Classification For Tools}.

\begin{table}
\centering
\begin{tabular}{c|c|l}
Actual Execution	& Tooling		& Result \\
\hline
sdep			& sdep			& Ordered \\
\cline{2-3}
			& $\neg$sdep		& False Positive (Warning?) \\
\hline
$\neg$sdep		& sdep			& False Negative (Warning?) \\
\cline{2-3}
			& $\neg$sdep		& Unordered \\
\end{tabular}
\caption{Dependency Classification For Tools}
\label{tab:Dependency Classification For Tools}
\end{table}

Any misclassification will result in either a false positive or
a false negative, except that many tools give some indication of a
misclassification, for example, printing a warning message indicating
that the code is too complex for it to analyze.
Some projects would choose to restructure the code in such cases, on the
grounds that code that is difficult for an automated tool to understand
is also likely to be misunderstood by its all-too-human developers.

One advantage that tools often have over C++ implementations is the
ability to devote much more computational power to the problem at hand.
In fact, some tools respond to excessive complexity by consuming all
available CPU and memory, which can also be interpreted as a good and
sufficient warning message.

However, there is an important special case for tools that are closely
associated with a C++ compiler.
Such tools can simply use that compiler's classification of
dependencies as semantic on the one hand or nonsemantic on the
other, whether by working with the compiler's intermediate
representations, examining binaries produced by the compiler, tracing
the program's execution, or, in the case of dynamic tools, actually
executing the program.
Either way, this approach reduces the problem of analysis from the
level of the C++ abstract machine to a much simpler lower level of
abstraction.
While taking this approach sacrifices significant generality, it also
has the significant benefit of greatly reducing the cost of dependency
analysis, potentially all the way down to zero.

\section{Future Directions}
\label{sec:Future Directions}

This paper focuses solely on compilers, but it should be possible to
extend these results to some classes of interpreters and JITs on the
one hand and to link-time optimizations (LTO) on the other.

This paper focuses primarily on compilers that do only local per-thread
analysis.
Further explorations could consider additional classes of global analysis,
for example, analyses that demonstrate that a given expression will
always evaluate to a particular constant.
JMM Causality Test Case 1 % %%% Cite P3064.
% %%% discussed in
% %%% Appendix~\ref{app:Causality Test Case 1}
provides an example of such a global analysis.

This paper assumes that the user defines threads, and that the
C++ implementation executes those threads unchanged.
Future analyses might examine the possibility of ``flattening''
optimizations that combine multiple threads into one.
(Note that such optimizations are more difficult than they might first appear.)

This paper shows that semantic dependencies are relative to specific
executions of a program and the specific compiler in use rather than
being determined solely by the source code.
Future work might expand on
Section~\ref{sec:Relative versus Absolute Dependency},
exploring special cases in which absolute semantic dependencies are
functions just of the source code.

This paper considers only programs whose threads
communicate using shared memory.
Future work might include the effects of input and output or
other operations which can vary from one execution to another,
such as those based on the current time, process IDs, or pointer values
(see Section~\ref{sec:Effect of Memory Layout}).

Finally, future work might expand on
Sections~\ref{sec:Global Analysis and Volatile versus Quasi Volatile},
and~\ref{sec:Merging Quasi-Volatile Loads}
so as to more precisely delineate the limits of permissible
behavior for quasi-volatile object accesses.

\section{Conclusion}
\label{sec:Conclusion}

This paper has presented samples of code containing non-trivial
semantic dependencies, uncovering some shortcomings in typical definitions of
``semantic dependency''.
It pointed out examples where semantic dependencies
are a function of an execution rather than of the source code, and examples
where a single semantic dependency extends from multiple loads
to a store but not from any one of those loads.
These complex dependencies motivated a generalization of the definition
of ``OOTA cycle'', which is shown in
Section~\ref{sec:Semantic Dependencies Can Be Many-To-One}.

This generalization, combined with a focus on compiler-based loose C++
implementations using single-thread analysis, and a consideration of
the relation between source code and machine code, led to the formulation
of the Fundamental Property of semantic dependency shown in
Section~\ref{sec:The Fundamental Property of Semantic Dependencies}.
This in turn led to a precise definition of ``semantic dependency'' given in
Section~\ref{sec:For Compilers Using Single-Thread Analysis}.
This definition was used in
Section~\ref{sec:Verifying the Fundamental Property}
to demonstrate that these implementations are
incapable of producing OOTA cycles in UB-free programs,
provided they treat all nonvolatile atomic objects as
quasi volatile, obeying the limitations outlined in
Section~\ref{sec:Volatile and Quasi Volatile Accesses}.

A variant definition of ``semantic dependency'' given in
Section~\ref{sec:For Compilers Using Global Analysis}, appropriate for
compiler-based loose C++ implementations that don't restrict
their code transformations to those based on single-thread analysis,
was used to show that such implementations
are incapable of producing OOTA cycles in UB-free programs
provided they treat all atomic objects as volatile.

This work shows that avoiding OOTA cycles requires no changes to the
standard, to current implementions, or to user practices for portable
code.
Future implementations must avoid optimizations that invent, duplicate,
or repurpose atomic loads.
More work is required to delineate any further restrictions on future
implementations that might carry out optimizations based on global
inter-thread analysis.

\clearpage
\appendix

% \section{But What About Tooling?}
% \label{sec:But What About Tooling?}
% @@@ State conclusions if needed where needed.

% \subsection{Load/Store Ordering: Hardware View for Software Hackers}
% \label{sec:Load/Store Ordering: Hardware View for Software Hackers}
% @@@ State conclusions if needed where needed.

% \subsection{Status Quo and Focused Tooling}
% \label{sec:Status Quo and Focused Tooling}
% @@@ State conclusions if needed where needed.

% \subsection{Change Relaxed to Forbid Load Buffering}
% \label{sec:Change Relaxed to Forbid Load Buffering}
% @@@ State conclusions if needed where needed.

% \subsection{Add Load-Store Memory Order that Forbids Load Buffering}
% \label{sec:Add Load-Store Memory Order that Forbids Load Buffering}
% @@@ State conclusions if needed where needed.

\section{Illustrative Litmus Tests}
\label{app:Illustrative Litmus Tests}

These litmus tests helped illuminate important aspects of the problem
of defining sdep and identifying OOTA cycles.
This appendix presents these tests in roughly decreasing order of
importance, surprise, and illumination.
It evaluates them using the \co{herd7} tool and also using manual
analysis.

\subsection{Semantic Dependencies and \co{volatile}}
\label{app:Semantic Dependencies and volatile}

The Linux kernel uses volatile accesses to constrain the compiler,
and it is worth looking at the example shown in
Listing~\ref{lst:Non-Volatile Accesses and Dependencies}
to see how this works.

\begin{listing}
\begin{verbatim}
 1 atomic<int> x, y;
 2
 3 void thread1()
 4 {
 5   int r1 = x.load(memory_order_relaxed);
 6   y.store(r1, memory_order_relaxed);
 7 }
\end{verbatim}
\caption{Non-Volatile Accesses and Dependencies}
\label{lst:Non-Volatile Accesses and Dependencies}
\end{listing}

Because both \co{x} and \co{y} are nonvolatile, a C++ compiler is
free to assume that the value loaded by line~5 will be either some value
stored to \co{x} or its initial value.
And because there are no stores to \co{x}, the only remaining possibility
is the default-initialized value of zero.
Therefore, the compiler is within its rights to substitute
the constant zero for \co{r1} on line~6, eliminating the semantic
dependency that would otherwise extend from line~5 to line~6.

\begin{listing}
\begin{verbatim}
 1 volatile atomic<int> x, y;
 2
 3 void thread1()
 4 {
 5   int r1 = x.load(memory_order_relaxed);
 6   y.store(r1, memory_order_relaxed);
 7 }
\end{verbatim}
\caption{Volatile Accesses and Dependencies}
\label{lst:Volatile Accesses and Dependencies}
\end{listing}

One way to preserve this dependency is use of \co{volatile}, as shown in
Listing~\ref{lst:Volatile Accesses and Dependencies}.
Now the compiler is forbidden from assuming that the value loaded
by line~5 has any relation to any stores to or initialization of \co{x}.

In this case, the \co{volatile} keyword alerts the compiler to
the possibility of interfering changes to \co{x} from outside the
program, for example, due to \co{x} being:
\begin{itemize}
\item	Allocated in an MMIO region of the address space.
\item	Modified by an unknown-to-the-compiler thread
	or signal handler.
\item	Subject to I/O-device DMA operations.
\item	Modified by debugger commands.
\item	Modified using facilities such as \path{/dev/mem}.
\item	Modified by as-yet-unwritten dynamically linked libraries.
\item	Modified by some other mechanism of the reader's choosing.
\end{itemize}
These possibilities force the compiler to preserve the semantic
dependency from line~4 to line~5,
and this preserved dependency helps prevent the formation of OOTA cycles
in environments where there is no interference.\footnote{
	When there is interference, the definition of ``OOTA cycle''
	does not apply.}

The developer is free to arrange for \co{x} to be free of any
interference, thus obtaining predictable behavior along with the
preserved semantic dependency.
However, this example invalidates the rough definition of semantic
dependency from
Section~\ref{sec:Simple OOTA Cycle}
because the value loaded from \co{x} will always be the initial value
of zero, so that it is at best dubious to talk about any changes in
the value loaded.

One way forward is to assume that interference could happen when
determining which dependencies are semantic, and then do further analysis
using that determination.
This forms the basis for the volatile approach presented in
Section~\ref{sec:Global Analysis and Volatile versus Quasi Volatile}.

\subsection{Non-Trivial Semantic Dependencies}
\label{app:Non-Trivial Semantic Dependencies}

Most of the examples in this paper involve simple semantic dependencies
connecting a single load to a single store,
but this section will present a more difficult example.

\begin{listing}[tbp]
@@ DisplayRunLitmus litmus/oota-mult-0.litmus @@
\caption{OOTA Multiplication Example, Initial Value Zero}
\label{lst:OOTA Multiplication Example, Initial Value Zero}
\end{listing}

Listing~\ref{lst:OOTA Multiplication Example, Initial Value Zero}
shows an example that builds on the multiplication-by-zero
discussion in
Section~\ref{sec:Semantic Dependencies and Source Code}
on
page~\pageref{sec:Semantic Dependencies and Source Code}.
Line~11 of \co{P0()} stores to \co{z} the product of the values loaded from
\co{x} and \co{y} on lines~9 and~10, respectively.
Similarly, lines~16 and~17 store to \co{x} and \co{y}, respectively,
the value that line~15 loads from \co{z}.
Because all three variables are initialized to zero, and because loads
can be ordered before any store, there is an execution in which all
three loads return the value zero, in which case all three stores will
of course store the value zero.
In any related execution where the value loaded from \co{y} remains zero,
nonzero values loaded from \co{x} cannot affect the value stored by
line~11 to \co{z}.
Only executions which load different (that is, nonzero) values from
both \co{x} and \co{y} can cause the value stored to \co{z} to change
from zero to a nonzero value.

Therefore, as noted in
Section~\ref{sec:Semantic Dependencies and Source Code},
in an execution where the loads from both \co{x} and \co{y} return zero,
there is no semantic dependency from the load from \co{x} to the
store to \co{z}, nor is there a semantic dependency from the load
from \co{y} to the store to \co{z}.
However, there \emph{is} a semantic dependency from the \emph{combination}
of the loads from \co{x} and \co{y} to the store to \co{z}.
An example is given by the execution that satisfies the
\co{exists} clause on line~20, which would be an OOTA cycle.

\begin{listing}[tbp]
@@ DisplayRunLitmus litmus/oota-mult-0-cond.litmus @@
\caption{OOTA Conditional Multiplication Example, Initial Value Zero}
\label{lst:OOTA Conditional Multiplication Example, Initial Value Zero}
\end{listing}

For further evidence that there is no semantic dependency from \co{y}
to \co{z} in executions where the load from \co{x} is zero, see the
equivalent\footnote{
	But equivalent only because the atomics are all nonvolatile!}
program shown in
Listing~\ref{lst:OOTA Conditional Multiplication Example, Initial Value Zero}.
Here, there is not even a load from \co{y} in executions where the
load from \co{x} returns zero, so there cannot possibly be a semantic
dependency involving this non-existent load.
A similar transformation would load \co{x} only in executions where the
load from \co{y} returned nonzero, which similarly illustrates the lack
of a semantic dependency between \co{x} and \co{z}.

In both
Listings~\ref{lst:OOTA Multiplication Example, Initial Value Zero}
and~\ref{lst:OOTA Conditional Multiplication Example, Initial Value Zero},
an OOTA cycle having non-zero values (such as those indicated by their
respective \co{exists} clauses) would need to affect both the \co{x}
and the \co{y} components of this compound semantic dependency.
This situation poses a challenge to current work on semantic dependencies,
which are typically restricted to a single load and store.

\begin{listing}[tbp]
@@ DisplayRunLitmus litmus/oota-mult-1.litmus @@
\caption{OOTA Multiplication Example, Initial Value 1}
\label{lst:OOTA Multiplication Example, Initial Value 1}
\end{listing}

In contrast, consider
Listing~\ref{lst:OOTA Multiplication Example, Initial Value 1},
which differs from
Listing~\ref{lst:OOTA Multiplication Example, Initial Value Zero}
only in the initial values (1 instead of 0) and the \co{exists}
clause (0 instead of 1).
Note that an execution satisfying the updated \co{exists} clause is again
an OOTA cycle.

Starting with the intuitive execution where the values loaded from \co{x}
and \co{y} on lines~9 and~10 are the value 1 (that is, nonzero), executions
that differ only in the value loaded from either \co{x} or \co{y} will
now affect the value stored to \co{z} by line~11.
There is therefore one semantic dependency from the load from \co{x}
to the store to \co{z} and another separate semantic dependency from
the load from \co{y} to the store to \co{z}, which matches typical
definitions of semantic dependency.
Yet if the initial values were instead obtained as input,
Listing~\ref{lst:OOTA Multiplication Example, Initial Value Zero}
and
Listing~\ref{lst:OOTA Multiplication Example, Initial Value 1}
could be thought of as being two different executions of the exact same
program.

\begin{listing}[tbp]
@@ DisplayRunLitmus litmus/oota-mult3-0.litmus @@
\caption{OOTA Three-Factor Multiplication Example, Initial Value Zero}
\label{lst:OOTA Three-Factor Multiplication Example, Initial Value Zero}
\end{listing}

Because the only two solutions to $x^2 = x$ are the values zero and one,
Listings~\ref{lst:OOTA Multiplication Example, Initial Value Zero}
and~\ref{lst:OOTA Multiplication Example, Initial Value 1}
each have only two OOTA cycles.
Note well that although software cannot distinguish the initial
values from those of an OOTA cycle having those same values, there is
a very real difference in the corresponding executions.
To wit, an execution not corresponding to an OOTA cycle will read at
least one initialization value, while an execution corresponding to an
OOTA cycle with those same values will read only from that cycle's stores.
On the other hand, in a similar program having initial values of (say)
two, both executions containing OOTA cycles would be distinguishable
from the initial values.

Listing~\ref{lst:OOTA Three-Factor Multiplication Example, Initial Value Zero}
goes one step further by multiplying the values obtained from three
different loads.
Because $x^3 = x$ has three solutions ($\left\{ -1, 0, 1 \right\}$),
this execution could have up to three OOTA cycles.

Much more elaborate examples can be constructed,
which raises the question of how to determine the set of semantic
dependencies in a given execution of a given fragment of code.
The short answer is that when running on a real-world system, 
any C++ program is a finite-state machine, which, unlike Turing-complete
systems, can be analyzed, as demonstrated in
Section~\ref{sec:Hardware Dependencies, Instruction Ordering, and the Fundamental Property}.

One might ask why semantic dependencies can be so complicated when hardware
dependencies are much more straightforward.
The reason is that, unlike compilers, hardware:
\begin{itemize}
\item	Respects dependencies even when they reduce to a constant,
\item	Optimizes much less aggressively, if at all, and
\item	Lacks undefined behavior.
\end{itemize}
This situation further underscores the per-execution nature of semantic
dependencies as well as the need for an improved definition of ``semantic
dependency''.
In
Section~\ref{sec:A Definition of Semantic Dependency},
we expand the traditional definition to include dependencies extending
from groups of loads to a single store.

\subsection{Why rfe Instead of Tried-And-True rf?}
\label{app:Why rfe Instead of Tried-And-True rf?}

Listings~\ref{lst:OOTA With an Intrathread Store-Load Link}
and~\ref{lst:OOTA With an Intrathread Store-Load Link, Optimized}
show the examples from
Section~\ref{sec:OOTA: rf versus rfe}
on
page~\pageref{sec:OOTA: rf versus rfe}
in the form of litmus tests.
A compiler is allowed to transform the program in
Listing~\ref{lst:OOTA With an Intrathread Store-Load Link}
to the form of
Listing~\ref{lst:OOTA With an Intrathread Store-Load Link, Optimized},
which demonstrates that intrathread reads-from (rfi) links may be
eliminated by the compiler and therefore should not be used in a
definition of OOTA.
This section provides a more involved example to drive the point home.

\begin{listing}[tbp]
@@ DisplayRunLitmus litmus/oota-3-2-proc.litmus @@
\caption{OOTA With an Intrathread Store-Load Link}
\label{lst:OOTA With an Intrathread Store-Load Link}
\end{listing}

\begin{listing}[tbp]
@@ DisplayRunLitmus litmus/oota-3-2-proc-opt.litmus @@
\caption{OOTA With an Intrathread Store-Load Link, Optimized}
\label{lst:OOTA With an Intrathread Store-Load Link, Optimized}
\end{listing}

The first step in this direction is
Listing~\ref{lst:Three-Process Version of JMM Causality Test Case 4},
which expands
Listing~\ref{lst:Causality Test Case 4}
% %%% Cite P3064.
% %%% (discussed in
% %%% Appendix~\ref{app:Causality Test Case 4})
from two threads to three.
The \co{herd7} tool reports OOTA values, and
all threads have straightforward semantic dependencies from their
loads to their stores.

\begin{listing}[tbp]
@@ DisplayRunLitmus litmus/oota-causality-4.litmus @@
\caption{Causality Test Case 4}
\label{lst:Causality Test Case 4}
\end{listing}

\begin{listing}[tbp]
@@ DisplayRunLitmus litmus/oota-3proc.litmus @@
\caption{Three-Process Version of JMM Causality Test Case 4}
\label{lst:Three-Process Version of JMM Causality Test Case 4}
\end{listing}

Listing~\ref{lst:Three-Process Version of JMM Causality Test Case 4 With Max and Min}
revises Listing~\ref{lst:Three-Process Version of JMM Causality Test Case 4}
so that the value stored to \co{y} is forced to be at most 17
and the value stored to \co{z} is forced to be at least 17.
This of course means that the value stored to \co{z} must always be
17, as can be seen in the \co{herd7} output.
(However, a single-thread analysis could not prove this fact.)

\begin{listing}[tbp]
@@ DisplayRunLitmus litmus/oota-whyrfe-3.litmus @@
\caption{Three-Process Version of JMM Causality Test Case 4 With Max and Min}
\label{lst:Three-Process Version of JMM Causality Test Case 4 With Max and Min}
\end{listing}

Listing~\ref{lst:Why rfe Instead of Tried-And-True rf?}
further revises the litmus test
to flatten \co{P0()} and \co{P1()} into a single thread, so that the
resulting \co{P0()} has an rfi link connecting
the store to and load from \co{y}, rather than the rfe link in
Listing~\ref{lst:Three-Process Version of JMM Causality Test Case 4 With Max and Min}.
As a result, even a single-thread analysis could conclude that the value
stored to \co{z} will always be 17, by examining \co{P0()} in isolation and
using the fact that it is valid for a C++ compiler to assume the value
loaded from \co{y} will always be the value just stored and so to omit
the load.\footnote{
	This is not an option in C because in that language,
	atomic accesses are volatile and hence must be preserved
	in the machine code.}

\begin{listing}[tbp]
@@ DisplayRunLitmus litmus/oota-whyrfe.litmus @@
\caption{Why rfe Instead of Tried-And-True rf?}
\label{lst:Why rfe Instead of Tried-And-True rf?}
\end{listing}

If a compiler makes this observation and optimizes the program by
storing a constant 17 to \co{z} rather than going through the
computations on lines~15--17 of
Listing~\ref{lst:Why rfe Instead of Tried-And-True rf?},
it will generate an executable which could produce the all-17's result
described by the \co{exists} clause.
(See Listing~\ref{lst:Why rfe Instead of Tried-And-True rf When z=17?},
where the store to \co{z} has been moved up to line~9.\footnote{
	The load from \co{y} on line~15 is retained even though it is
	not used, because \co{r3} appears in the \co{exists} clause and
	hence its final value is considered observable behavior.})
This would not be an OOTA cycle, because in either form of the program
the store to~\co{z} in \co{P0} is not semantically dependent on the
load from~\co{x}, so there is no cycle in (sdep $\cup$ rfe).
A different compiler that does not make this optimization will
generate an executable which cannot produce the all-17's result at all,
because of hardware dependencies and consequent instruction ordering
in the machine code.

Either way, no OOTA cycle can occur.
But what if we chose to define an OOTA cycle with (sdep $\cup$ rf) instead
of (sdep $\cup$ rfe)?

\begin{listing}[tbp]
@@ DisplayRunLitmus litmus/oota-whyrfe-z17.litmus @@
\caption{Why rfe Instead of Tried-And-True rf When z=17?}
\label{lst:Why rfe Instead of Tried-And-True rf When z=17?}
\end{listing}

In that case, it would be necessary to consider separately the dependency
in Listing~\ref{lst:Why rfe Instead of Tried-And-True rf?}
connecting \co{P0()}'s load from \co{x} to its store to \co{y},
and that connecting its load from \co{y} to its store to \co{z}.
The first is a genuine semantic dependency; the second might or might
not be, depending on the implementation.
A compiler that uses single-thread analysis and does not omit the load
from \co{y} would conclude that it is, because of the
possibility that the value loaded from \co{y} might be larger than 17.
From this point of view, the all-17's result \emph{would} be
considered an OOTA cycle.

This example demonstrates another reason why defining OOTA cycles
in terms of (sdep $\cup$ rf) is problematic.
Doing so can lead to classifying a cycle as OOTA even though there
is no genuine semantic dependency in one of the participating threads.

\subsection{Inventing Atomic Loads}
\label{app:Inventing Atomic Loads}

This section contains several example litmus tests
expanding on the material presented in
Section~\ref{sec:Inventing Atomic Loads Can Destroy Semantic Dependencies}.
on
page~\pageref{sec:Inventing Atomic Loads Can Destroy Semantic Dependencies}.

\subsubsection{Illustrative Atomic-Load Invention}
\label{app:Illustrative Atomic-Load Invention}

Consider Listing~\ref{lst:No Invented Atomic Loads}.
At first glance, it might appear straightforward.
\co{P0()}'s store to \co{z} clearly depends on its load from \co{y},
and \co{P1()}'s store to \co{y} clearly depends on its load from \co{z}.
If both dependencies are semantic, the OOTA cycle cannot be realized.

\begin{listing}[tbp]
@@ DisplayRunLitmus litmus/oota-no-invented-load.litmus @@
\caption{No Invented Atomic Loads}
\label{lst:No Invented Atomic Loads}
\end{listing}

\begin{listing}[tbp]
@@ DisplayRunLitmus litmus/oota-load-invented.litmus @@
\caption{Atomic Loads Invented}
\label{lst:Atomic Loads Invented}
\end{listing}

But is \co{P0()}'s dependency truly semantic?
Imagine what might happen if a C++ compiler is permitted to
duplicate \co{P0()}'s load from \co{x} and then make use of both loads'
values.
This would allow a compiler to transform
Listing~\ref{lst:No Invented Atomic Loads}
into
Listing~\ref{lst:Atomic Loads Invented}.

Because of the comparisons against zero on lines~9 and~10 of
Listing~\ref{lst:No Invented Atomic Loads},
the values of \co{r1} and \co{r2} are known to be either zero or one.
Hence if the values of \co{r1a} and \co{r1b} on lines~9 and~10 of
Listing~\ref{lst:Atomic Loads Invented}
differ (say because \co{P2()} changes the value of \co{x}
during the time between the two loads)
then one of them must be equal to the value that would
be obtained for \co{r2}.
If the compiler is further permitted to make a choice at runtime
between the two values loaded from \co{x}, it could always choose
to use for \co{r1} the one that would be equal to \co{r2} (see line~16).
Then the value stored to \co{z} would always be one,
as shown on line~14, where the store has been moved before
the load from \co{y} on line~15 to emphasize the fact that
it does not depend on that load.
The dependency from~\co{y} to~\co{z} would not be semantic after all,
and the outcome \co{r1 == r2 == 1} would indeed be possible
(although it would not be an example of OOTA because of the lack of a
semantic dependency).

This is a simple example of a more general phenomenon.
Suppose $z$ is given by some function $f$ of $x$ and $y$.
Under what conditions can we say that the value of $z$ doesn't depend
on $y$?
The standard answer is that this happens when there are values
$z_0$ and $x_0$ such that for any $y$, we have $z_0 = f(x_0, y)$.
But if we are allowed to choose from among multiple values of $x$,
the situation gets more complicated.
Then the answer would be that there is a value $z_0$ and
a set $X$ of values for $x$ such that for any $y$, there is some
$x \in X$ with $z_0 = f(x, y)$.
In Listing~\ref{lst:Atomic Loads Invented},
$f$ is the equality function and $X$ is simply the set $\{0, 1\}$;
however, the reasoning applies in any situation where $x$ ranges
over a finite collection of possible values.
And as the example shows, the more general condition can hold in situations
where the simpler condition does not, indicating that our intuitive
notions of semantic dependency are not adequate when there can be
multiple loads of the same variable.

\begin{listing}[tbp]
@@ DisplayRunLitmus litmus/oota-unused-load.litmus @@
\caption{Unused Extra Atomic Load}
\label{lst:Unused Extra Atomic Load}
\end{listing}

\subsubsection{Illustrative Atomic-Load Use Invention}
\label{app:Illustrative Atomic-Load Use Invention}

Clearly something has gone wrong if a valid (in loose C++) transformation
like this one is capable of destroying what should be an obvious semantic
dependency.
One possible reaction is to declare that useful C++ compilers should never
invent or duplicate nonvolatile relaxed atomic loads.
But what if the initial code was as shown in
Listing~\ref{lst:Unused Extra Atomic Load}?\footnote{
	Unused loads can easily be generated by complex macros or
	template metaprograms.
	Typically compilers then remove them, but they are not obliged to.}
In this case, the transformation to
Listing~\ref{lst:Atomic Loads Invented}
would not require inventing or duplicating a load,
but instead merely inventing a use for
the previously discarded value in \co{r0}.

There are a number of possible reactions to these situations:

\begin{enumerate}
\item	Listing~\ref{lst:No Invented Atomic Loads}
	looks like it has an OOTA cycle, but there is in fact no
	semantic dependency.
\item	Attempts to use real-world constraints to prevent OOTA cycles
	are futile.
\item	Compilers should not invent uses for values from
	nonvolatile relaxed atomic loads.
	If it turns out that there are useful optimizations that invent
	such uses, then such optimizations must be applied only with
	careful attention to the as-if rule.
	Some might argue that inventing comparisons between a pair of
	nonvolatile relaxed atomic loads from the same object should
	be permitted only if the two loads could be reordered to be
	adjacent to each other.
\item	Any time a compiler might be tempted to invent uses for
	values of nonvolatile relaxed atomic loads from the same
	object, the loads should instead be merged into a single load.
	This reaction might be attractive to those who carefully consider
	the pointlessness of keeping two loads that return the same value
	on the one hand or the cache-miss overhead incurred when closely
	spaced loads return different values on the other.
\item	Compilers that invent nonvolatile atomic loads or invent
	new uses for nonvolatile atomic loads can adversely affect
	observed behavior, for example, by introducing error into code
	attempting to do statistical sampling.
\item	What other interesting situations might come to light?
\end{enumerate}

However, before making such a choice, we should look at other situations.

\subsubsection{Customary Non-Atomic-Load ``Invention''}
\label{app:Customary Non-Atomic-Load Invention}

There are situations where inventing loads from non-atomic objects
is customary, for example, when hoisting a load out of a loop.
In this case, compilers might hoist the load from \co{x} out of
the loop:
\begin{quote}
\begin{verbatim}
do_something_with(x);
for (i = 0; i < limit; i++)
    y[i] += x;
\end{verbatim}
\end{quote}
A compiler might act as if the program had instead been this:
\begin{quote}
\begin{verbatim}
r1 = x;
do_something_with(r1);
for (i = 0; i < limit; i++)
    y[i] += r1;
\end{verbatim}
\end{quote}
Note that passing \co{x} to \co{do_something_with()} is important
when \co{x} is not atomic,
because this allows the compiler to assume that inventing additional
accesses to \co{x} will not result in a data race.
Without this assumption, the above transformation might introduce a data
race when \co{limit} is less than or equal to zero.
Hence the use of quotation marks in the section heading:
Here the load is not actually invented, but instead merged with an
earlier load such that there is no synchronization between the two loads.

But when \co{x} is a nonvolatile atomic, there are no worries about
data races.
In theory, if the \co{do_something_with()} call was not present
the compiler could still make the transformation above with no
qualms, especially given that the value loaded from \co{x} is discarded in
the case where \co{limit} is less than or equal to zero.
Except that now the data-race worry is replaced by a cache-miss worry;
after all, optimizations are supposed to speed things up,
not slow them down.
A compiler might be wise to adhere to the
same restriction that avoids data races for loads from non-atomic
variables in order to avoid potentially expensive cache misses.

In addition, compiler writers should carefully consider the issues
raised by the following section.

\subsubsection{Other Cautions for Atomic-Load Invention}
\label{app:Other Cautions for Atomic-Load Invention}

Consider the code below:
\begin{quote}
\begin{verbatim}
int r0 = x;
int r1 = r0 * r0 + 2 * r0 + 1;
\end{verbatim}
\end{quote}
For sufficiently small values of \co{x}, the abstract machine is
guaranteed to produce a perfect square in \co{r1}.

But if the compiler is permitted to invent atomic loads then the compiled
code might not respect this guarantee, thus incorrectly deviating
from the abstract machine.
This issue is evident in a transformed version of the above code:
\begin{quote}
\begin{verbatim}
int r0 = x;
int invented = x;
int r1 = r0 * r0 + 2 * invented + 1;
\end{verbatim}
\end{quote}
For example, if the load into \co{r0} happened when the value of \co{x}
was zero and the load into \co{invented} happened when the value of
\co{x} was 10, then the value of \co{r1} will be 21, which is not
a perfect square.

Similar issues can arise even when the compiler merely repurposes a
value from an atomic load, as can be seen from this example:
\begin{quote}
\begin{verbatim}
do_something(x);
int r0 = x;
int r1 = r0 * r0 + 2 * r0 + 1;
\end{verbatim}
\end{quote}
Again, for sufficiently small values of \co{x}, the abstract machine is
guaranteed to produce a perfect square in \co{r1}.
So is the compiler allowed to transform this example into the following?
\begin{quote}
\begin{verbatim}
repurposed = x;
do_something(repurposed);
int r0 = x;
int r1 = r0 * r0 + 2 * repurposed + 1;
\end{verbatim}
\end{quote}
This can result in the value of \co{r1} being something other than
a perfect square, which is incorrect behavior.
%
But it would be just fine if the compiler fully merged the two loads from
\co{x}, for example, as follows:
\begin{quote}
\begin{verbatim}
int r0 = x;
do_something(r0);
int r1 = r0 * r0 + 2 * r0 + 1;
\end{verbatim}
\end{quote}

Similar observations apply to atomic loads from pointers.
For example, consider
Listing~\ref{lst:Example Duplicated Load} on
page~\pageref{lst:Example Duplicated Load},
in which the abstract machine guarantees that line~17 passes two
fields from the same structure to \co{do_something_with()}.
This guarantee will not always be met if the compiler is permitted to
invent atomic loads such as that contained in the commented-out line~15.
\begin{listing}[tbp]
\scriptsize
\begin{verbatim}
 1 struct foo {
 2   int a;
 3   int b;
 4 };
 5 _Atomic struct foo *globalfoop;
 6
 7 void bar()
 8 {
 9   int a;
10   int b;
11   struct foo *fp;
12
13   fp = atomic_load_explicit(&globalfoop, memory_order_acquire);
14   do_something(fp);
15   fp = atomic_load_explicit(&globalfoop, memory_order_acquire);
16   a = fp->a;
17   // fp = atomic_load_explicit(&globalfoop, memory_order_acquire);
18   b = fp->b;
19   do_something_with(a, b);
20 }
\end{verbatim}
\caption{Example Merging of Duplicated Loads}
\label{lst:Example Merging of Duplicated Loads}
\end{listing}
But suppose that there was an earlier load from \co{globalfoop}, as
shown on lines~13 and~14 in
Listing~\ref{lst:Example Merging of Duplicated Loads}.
Then, assuming that \co{do_something()} is synchronization-free, the
second load on line~15 of that listing could be removed without resulting
in incorrect behavior.

Please note that these examples are not just accidents of the current state
of the C++ abstract machine.
After all, it is imperative that users be able to rely on the laws of
arithmetic and on atomic accesses to pointers.
Therefore, compilers must refrain from inventing atomic loads and from repurposing
their values.
However, compilers are permitted to merge adjacent atomic loads from
the same variable.
These restrictions correspond to our notion of quasi volatile defined in
Section~\ref{sec:Volatile and Quasi Volatile Accesses}.

\subsection{Undefined Behavior and Unwise Optimization}
\label{app:Undefined Behavior and Unwise Optimization}

Many compilers carry out optimizations based on the assumption
that undefined behavior (UB) cannot happen,
and if not performed carefully, these optimizations may be invalid
because of OOTA-like interactions.
A simple example is shown in
Listing~\ref{lst:OOTA-Like Behavior Due To Divide-By-Zero UB}.

\begin{listing}[tbp]
@@ DisplayRunLitmus litmus/oota-div-ub.litmus @@
\caption{OOTA-Like Behavior Due To Divide-By-Zero UB}
\label{lst:OOTA-Like Behavior Due To Divide-By-Zero UB}
\end{listing}

On line~14 of this listing, the computation \co{1 / (r2 <= 0)}
will yield one if \co{r2} is not positive and will result in UB
otherwise (a divide-by-zero error).
It may not be obvious, but the program is in fact UB-free---there are
no abstract executions in which \co{r2} is greater than zero.
To see why not, consider that in any such execution
\co{P1()} would not store anything to \co{y};
it certainly would not store a positive value.
This means that the value loaded on line~8 would have to be \co{y}'s
initial value of zero, and so would the value stored to \co{x} on
line~9, and hence it would not be possible for the load on line~13
to get a positive value for \co{r2} in the first place.

Even for a compiler analyzing \co{P1()} in isolation,
it is clear that either line~14 will store one to \co{y} or
else the division by zero will cause UB.
This allows the compiler to assume that the store will always
take place, on the grounds that UB cannot happen,
and so the compiler may optimize the comparison and division
by replacing the whole expression with a constant \co{1}.
There will then be no dependency from the load on line~13
to the store and thus nothing forcing the store to execute
after the load.

When the resulting program runs on a weakly ordered architecture,
the execution could go as follows:
\begin{enumerate}
\item	Line~14 executes out of order, storing the constant \co{1} to \co{y}.
\item	Line~8 loads one from \co{y}, setting \co{r1} to one.
\item	Line~9 stores one to \co{x}.
\item	Line~13 loads one from \co{x}, setting \co{r2} to one.
\end{enumerate}
The final result would satisfy the \co{exists} clause,
exhibiting observable behavior (\co{r2} is one at the end)
that no abstract execution of the original program could produce.
(This could happen even if all of the atomic objects in
Listing~\ref{lst:OOTA-Like Behavior Due To Divide-By-Zero UB}
were converted to volatile.)

This result shows that the proposed optimization was invalid.
To prevent the unwanted behavior, the compiler would have to take an
extra step to force the optimized store on line~14 to be ordered after
the load on line~13, possibly by changing the load to a load-acquire
or changing the store to a store-release.
Hans Boehm has proposed this store-release alternative as his
Interpretation~B'~\cite{HansBoehm2020UBalternatives}.
% @@@ Do we want to add a disclaimer about completeness or lack thereof
% @@@ of this solution?

In other words, the compiler would have to preserve the dependency-induced
ordering that the UB-based optimization would otherwise destroy.
It is tempting to consider using techniques
% Appendix~\ref{app:Aside on Undefined Behavior}
that prevent back-propagation of UB, but these are ineffective in this
case because the compiler can detect and exploit UB when
considering line~14 in isolation.

\clearpage

\bibliographystyle{plain}
\addcontentsline{toc}{section}{References}
\bibliography{bib/RCU,bib/WFS,bib/hw,bib/os,bib/parallelsys,bib/patterns,bib/perfmeas,bib/refs,bib/syncrefs,bib/search,bib/swtools,bib/realtime,bib/TM,bib/standards,bib/memorymodel.bib}

\end{document}
