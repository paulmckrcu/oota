\documentclass[10]{article}

% standard packages

% A more pleasant font
\usepackage[T1]{fontenc} % use postscript type 1 fonts
\usepackage{textcomp} % use symbols in TS1 encoding
\usepackage{mathptmx,helvet,courier} % use nice, standard fonts for roman, sans and monospace respectively

% Improves the text layout
\usepackage{microtype}

\usepackage{lscape}
\usepackage{fancyhdr}
\usepackage{epsfig}
\usepackage{subfigure}
\usepackage{url}
\usepackage{graphics}
\usepackage{enumerate}
\usepackage{ifthen}
\usepackage{float}
\usepackage{listings}
\lstset{basicstyle=\ttfamily}
% \usepackage[strings]{underscore}
% \usepackage{underscore}
\usepackage[bookmarks=true,bookmarksnumbered=true,pdfborder={0 0 0}]{hyperref}

\lstset{
  literate={\_}{}{0\discretionary{\_}{}{\_}}%
}

\usepackage[table]{xcolor}
\usepackage{booktabs}

\DeclareUrlCommand\email{}

\pagestyle{fancy}
\rhead{}

\newfloat{listing}{tbp}{lol}
\floatname{listing}{Listing}

\begin{document}
\title{DNNNNR0: OOTA Execution is Provably Vacuous}

\newcommand{\co}[1]{\lstinline[breaklines=yes,breakatwhitespace=yes]{#1}}

\author{
Paul E.~McKenney\\\email{paulmck@kernel.org} \and
The Indefatigible TBD
}
\date{November 27, 2023 (Pre-Tokyo)}
\maketitle{}

Audience: SG1

\begin{abstract}
	The out-of-thin-air (OOTA) properties of the specification
	of \co{memory_order_relaxed} have resulted in considerable
	consternation over the years.
	Attempts to create memory models that rule out OOTA behaviors
	have either been non-executable, complex, or unloved by compiler
	writers.
	But at the same time, there are no known instances of OOTA
	behavior in real C++ implementations.

	This paper goes further, adding temporal constraints from
	long-standing laws of physics, thereby providing an informal
	proof that OOTA cannot occur in correctly constructed C++
	programs running on real-world systems.
	In other words, a correct C++ implementation necessarily
	preserves the semantic dependencies that prevent OOTA
	cycles from forming.
\end{abstract}

\section{Background}
\label{sec:Background}

\subsection{Brief OOTA Overview}
\label{sec:Brief OOTA Overview}

OOTA occurs when a group of threads load from each others' stores,
but where each thread's store depends on the value returned by that
thread's load.
A given out-of-thin-air value passes around the resulting cycle.
Because there is a cycle, on a real system, at least one of the links
must be counter-temporal, that is, either a load returns a value before
that value is stored, or a store's address or value is computed before
a prior load returns a value used in that computation.

\begin{listing}[tbp]
\begin{verbatim}
 1 int main() {
 2   atomic_int x=0;
 3   atomic_int y=0;
 4   int r1;
 5   int r2;
 6   {{{ { r1=x.load(memory_order_relaxed);
 7         y.store(r1,memory_order_relaxed); }
 8   ||| { r2=y.load(memory_order_relaxed);
 9         x.store(r2,memory_order_relaxed); }  }}}
10   return 0;
11 }
\end{verbatim}
\caption{OOTA Cycle}
\label{lst:OOTA Cycle}
\end{listing}

Listing~\ref{lst:OOTA Cycle}
shows an OOTA cycle, where the \co{\{\{\{} on line~5 begins a set of
threads, the \co{|||} on line~8 separates a pair of threads, and
the \co{\}\}\}} on line~9 ends a set of threads.
Lines~6 and~7 do a relaxed copy from \co{x} to \co{y}, and
lines~8 and~9 to a relaxed copy from \co{y} to \co{x}.
Even though both \co{x} and \co{y} are initialized to zero on
lines~2 and~3, respectively, according to the mathematical core
of the C++ memory model, the only constraint on the final values
of \co{x} and \co{y} is that they be equal.
Counterintuitivily, their final values could be any member of their type.

Note that C++ implementations are permitted to evaluate to a more
general notion of OOTA values in cases involving unspecified or undefined
behavior, such as use of uninitialized objects.
However, these situations do not involve OOTA cycles, but rather user
errors and other issues that can lead to unfriendly optimizations that
in turn result in unpredictable output.
This paper focuses instead on OOTA cycles.

C++ implementations may of course avail themselves of the as-if rule.
However, correct use of the as-if rule requires that the observable
behaviors of the program be appropriately preserved.

\subsection{Prior Work}
\label{sec:Prior Work}

Some executable C++ memory models correctly flagged executions involving
OOTA~\cite{JadeAlglave2014HerdingCats}.\footnote{
	Others cleverly avoid this issue by forbidding atomic
	stores of non-constant values~\cite{MarkBatty2011cppmem}.}
% @@@ Better herd7 C11 citation?
However, because these models are atemporal, they cannot reject
OOTA executions other than by flagging the OOTA value as arbitrary
(which they in fact do).

P0442R0 (``Out-of-Thin-Air Execution is Vacuous'')~\cite{PaulEMcKenney2016OOTA}
provided a decision procedure for classifying behaviors as permitted
misordering on the one hand or disallowed OOTA on the other, using
a perturbation method based on the insight that all OOTA behaviors are
fixed-point computations.

Some workers recommend avoiding OOTA by ordering prior relaxed
loads before subsequent relaxed
stores~\cite{Boehm:2014:OGA:2618128.2618134,HansBoehm2019OOTArevisitedAgain,Lahav:2017:RSC:3062341.3062352},
but this requires real instructions be executed, consuming real
time and real electrical power to solve a strictly theoretical
problem.

Other workers recommend various procedures to identify and avoid OOTA
cycles~\cite{Lahav:2017:RSC:3062341.3062352,Sinclair:2017:CAR:3079856.3080206,Lee:10.1145/3385412.3386010,MarkBatty2019ModularRelaxedDependenciesOOTA},
but none of these have been looked upon favorably by compiler implementers.

All this work focused on either identifying OOTA or on how C++
implementations could avoid it.
None of this work applied real-world temporal constraints to the OOTA
problem.
Which might explain why no known real-world C++ implementation results
in OOTA executions.

This paper therefore drops the question of how OOTA can be avoided and
instead focuses on proving that OOTA cannot occur.

\section{Real-World Constraints}
\label{sec:Real-World Constraints}

Real-world constraints are imposed by the standard, which have been
considered in prior work, and by the laws of physics, which have
not.

\subsection{Constraints of the Standard}
\label{sec:Constraints of the Standard}

Because C++ volatile atomics constitute observable behavior,\footnote{
	Working Draft, Standard for Programming Language C++
	\co{[intro.abstract]}.}
they must be executed in strict accordance with the rules of the abstract
machine.
This means that any C++ volatile atomic operation involving a store
must execute as if all loads whose return values are used to compute
that store's address or value have already returned.
OOTA afficienados will recognize this as a special case of ordering
relaxed loads before relaxed stores, albeit one not requiring
expensive memory-fence instructions on weakly ordered architectures.

Although non-volatile accesses to atomic objects are not observable behavior,
they are still subject to the memory model.
In particular, implementations cannot:

\begin{enumerate}
\item	Invent atomic stores.
\item	Duplicate atomic stores.
\end{enumerate}

\paragraph{Invented Atomic Stores}

\begin{listing}[tbp]
\begin{verbatim}
 1 int main() {
 2   atomic_int x = 2;
 3   volatile int r1;
 4   {{{ { x.store(42, memory_order_relaxed); // invented store
 5         x.store(3, memory_order_relaxed); }
 6   ||| r1 = x.load(memory_order_relaxed);
 7   }}};
 8   return 0;
 9 }
\end{verbatim}
\caption{CPPMEM Invented Store}
\label{lst:CPPMEM Invented Store}
\end{listing}

The reason that atomic stores cannot be invented is that doing so can
introduce new (and almost certainly undesireable) behaviors.
To see this, consider Listing~\ref{lst:CPPMEM Invented Store},
a CPPMEM\footnote{
	\url{http://svr-pes20-cppmem.cl.cam.ac.uk/cppmem/index.html}.}
litmus test that demonstrates such a behavior.
Without line~4, only the values~2 and~3 can be stored to \co{r1}.
With that line, the additional value~42 can also be stored to \co{r1}.
The compiler is therefore forbidded from inventing that store of 42
unless it can prove that doing so does not negatively affect the
program's observable behaviors, independently of OOTA cycles.
For example, the implementation might be able to prove that there
are no other accesses to \co{x} at the time of the invented store.

\paragraph{Duplicated Atomic Stores}

\begin{listing}[tbp]
\begin{verbatim}
 1 int main() {
 2   atomic_int x = 0;
 3   volatile int r1;
 4   {{{ { x.store(1, memory_order_relaxed); // duplicated store
 5         x.store(1, memory_order_relaxed); }
 6   ||| { r1 = x.load(memory_order_relaxed);
 7         x.store(2, memory_order_relaxed); }
 8   }}};
 9   return 0;
10 }
\end{verbatim}
\caption{CPPMEM Duplicated Store}
\label{lst:CPPMEM Duplicated Store}
\end{listing}

Duplicating atomic stores can also introduce new and undesireable
behaviors.
To see this, consider Listing~\ref{lst:CPPMEM Duplicated Store}, a CPPMEM
litmus test that demonstrates such a behavior.
Without line~4, if the final value of \co{r1} is 1, then the final value
of \co{x} must be 2.
With that line, the final value of \co{x} can be 1 even when the
final value of \co{r1} is 1.
The compiler is therefore forbidded from duplicating that store of 1
unless it can prove that doing so does not negatively affect the
program's observable behaviors, independently of OOTA cycles.
Again, the implementation might be able to prove that there
are no other accesses to \co{x} at the time of the invented store.

\paragraph{Omitted Atomic Stores}
In contrast, an implementation is permitted to omit atomic stores
under less restricted circumstances.
For example, a pair of back-to-back stores to \co{x} might always be
executed such that no other thread accesses \co{x} during the time
between those two stores.
This means that, short of inspecting the assembly code, the user has no
way of proving that the store was in fact omitted.
The user can always resort to volatile atomic stores or inline
assembly\footnote{
	For example, by placing the Linux-kernel \co{barrier()} macro
	between the two stores.
	This macro is an empty GCC \co{asm} that specifies the \co{memory}
	clobber.}
to prevent the compiler from omitting a store.
However, omitting a store cannot create an OOTA cycle.

\subsection{Laws of Physics}
\label{sec:Laws of Physics}

The speed of light is finite~\cite{OleRoemer1671SpeedOfLight}
and atoms are of non-zero size~\cite{JeanBaptistePerrin1923AtomSize}.
These laws of physics mean that if one thread loads the value
stored by some other thread, that load must have occurred later
in global time than did the store.
Current memory models do not capture these temporal constraints,
which makes it more difficult for them to efficiently rule out
OOTA behavior.

\section{How to Form an OOTA Cycle?}
\label{sec:How to Form an OOTA Cycle?}

Rather than look at how to prevent an implementation from forming
an OOTA cycle, this section instead looks at how to form one.
This will lead to the conclusion that a correct C++ implementation
cannot form an OOTA cycle.

Again, an OOTA cycle consists of loads from other threads' stores where
a given store's address or value depends on the value returned by
the corresponding thread's prior load.
Courtesy of the laws of physics, a given store must execute earlier
in global time than any load from that stored-to object that returns
the value stored.

And again, an OOTA cycle must have at least one link that goes backwards
in time.
The only possible candidate counter-temporal links so are those
confined to a given thread.
This in turn requires that a store that depends on an earlier load
be executed earlier in time than that load.
The two ways that this can happen are when the implementation:

\begin{enumerate}
\item	Is able to determine (prove) the location and value of the store.
\item	Guesses the location and value of the store and has some means
	to take corrective action should any guess prove incorrect.
\end{enumerate}

Each of these possibilities is discussed in the next sections.
followed by a section on the possibility of flattening a multithreaded
program into a single thread.

\subsection{Implementation Proves Value}
\label{sec:Implementation Proves Value}

Implementation can sometimes prove the location and value of a given
store.
For example, if the value loaded is multiplied by zero, the result
is known to be zero, so that the store can proceed prior to the load.
But in this case, there is no perturbation of the value returned from
the load that can affect the value stored.
Therefore, by P0442R0, that store cannot possibly take part in an
OOTA cycle.

\subsection{Implementation Guesses at Value}
\label{sec:Implementation Guesses at Value}

Hardware speculative execution is commonplace, and permits the
hardware to guess at values, squashing the speculation if any
of the guesses prove incorrect.
But in this case, a given store is not committed until all guesses that
this store depends on have been confirmed, and thus no uncommitted store
is visible to any other thread.
Therefore, hardware speculation cannot result in OOTA cycles, and
by design, even on weakly ordered
systems~\cite{ARMv7A:2010,ARMv8A:2017,PowerISA2.07-2013}.
This same constraint applies to software, which might use compare-and-swap
loops or hardware transactional memory to confirm guesses.

Either way, the need to wait for guesses to be confirmed before committing
stores prevents those stores from becoming visible prior to the
completion of any loads that those stores depend on.

\subsection{But What About Flattening?}
\label{sec:But What About Flattening?}

The concept of flattening multiple threads into a single thread is
tantalizing, especially given the large cache-miss penalties inherent
to large multicore systems, which can have many sockets containing
thousands of CPUs.
One might object to the whole concept of flattening as a compiler
optimization given the difficulty inherent in avoiding deadlocks
and other hazards.
Nevertheless, it is only reasonable to ask whether flattening can
result in OOTA cycles.
As we will see, the answer is ``no''.

But for the purposes of OOTA cycles, the flattening situation is
much simpler.
Any fully flattened program will be single threaded, which implies that
each of its atomic load operations will return the last value stored to
that same object in sequenced-before order (or the initial value if there
is no such store).
Thus, any sequence of loads and stores must have a first operation and
a last operation, and the last operation cannot affect the first
operation.
Therefore, cycles cannot possibly form in a flattened program,
which in turn implies that OOTA cycles cannot form.

\section{Correct Implementations Cannot Form OOTA Cycles}
\label{sec:Correct Implementations Cannot Form OOTA Cycles}

This section has shown that an implementation that correctly
handles the memory model, observable behavior, and the as-if rule
cannot form OOTA cycles.

\section{Conclusion}
\label{sec:Conclusion}

This paper has added temporal reasoning to OOTA analysis of the C++
memory model.
This work shows that correct C++ implementations are inherently incapable
of producing OOTA cycles.

% \section{History}
% \label{sec:History}

\bibliographystyle{plain}
\bibliography{bib/RCU,bib/WFS,bib/hw,bib/os,bib/parallelsys,bib/patterns,bib/perfmeas,bib/refs,bib/syncrefs,bib/search,bib/swtools,bib/realtime,bib/TM,bib/standards,bib/memorymodel.bib}

\end{document}
